%\subsection{HTTP Public Key Pinning (HPKP)}
Much like HTTP Strict Transport Security (HSTS), HTTP Public Key Pinning (HPKP) is a Trust On First Use (TOFU) mechanism. It protects HTTPS websites from impersonation using certificates issued by compromised certificate authorities. The data for Pinning is supplied by an HTTP-Header sent by the WebServer. 

\subsubsection{HPKP Header Directives}
\label{subsubsection:HPKP Header Directives}
HPKP provides two different types of headers:
\begin{itemize*}
  \item Public-Key-Pins
	\item Public-Key-Pins-Report-Only
\end{itemize*}
HPKP header can be parametrized by following directives:
\begin{itemize*}
  \item pin-sha256="<YOUR\_PUBLICKEY\_HASH=>"
  \item max-age=<number-of-seconds> 
	\item includeSubdomains 
	\item report-uri="<https://YOUR.URL/TO-REPORT>"
\end{itemize*}

\textbf{pin-sha256} is a required directive. It can and should be used several (at least two) times for specifying the public keys of your domain-certificates or CA-certificates. Operators can pin any one or more of the public keys in the certificate-chain, and indeed must pin to issuers not in the chain (as, for example, a Backup Pin). Pinning to an intermediate issuer, or even to a trust anchor or root, still significantly reduces the number of issuers who can issue end-entity certificates for the Known Pinned Host, while still giving that host flexibility to change keys without a disruption of service. OpenSSL can be used to convert the Public Key of an X509-Certificate as follows:
\begin{lstlisting}
openssl x509 -in <certificate.cer> -pubkey -noout | 
 openssl rsa -pubin -outform der | 
 openssl dgst -sha256 -binary | 
 openssl enc -base64
writing RSA key
pG3WsstDsfMkRdF3hBClXRKYxxKUJIOu8DwabG8MFrU=
\end{lstlisting} 
This piped usage of OpenSSL first gets the Public-Key of <certificate.cer>, converts it do DER (binary) format, calculates an SHA256 Hash and finally encodes it Base64. The output (including the ending Equal-Sign) is exactly whats needed for the \emph{pin-sha256="<YOUR\_PUBLICKEY\_HASH=>"} parameter. To generate the Hash for a prepared Backup-Key just create a Certificate-Signing-Request and replace "\texttt{openssl x509}" by "\texttt{openssl req -in <backup-cert.csr> -pubkey -noout}" as first OpenSSL command.\\ 
Instead of using OpenSSL even WebServices like \url{https://report-uri.io/home/pkp_hash/} can be used to get a suggestion for the possible Public-Key-Hashes for a given WebSite.

\textbf{max-age} is a required directive (when using the \emph{Public-Key-Pins} header). This directive specifies the number of seconds during which the user agent should regard the host (from whom the message was received) as a Known Pinned Host.

\textbf{includeSubdomains} is an optional directive. This directive indicates that the same pinning applies to this Host as well as \emph{any subdomains of the host's domain name}. Be careful - you need to use a Multi-Domain/Wildcard-Certificate or use the same Pub/Private-Keypair in all Subdomain-Certificates or need to pin to CA-Certificates signing all your Subdomain-Certificates.

\textbf{report-uri} is an optional directive. The presence of a report-uri directive indicates to the Browser that in the event of Pin Validation failure it should post a report to the report-uri (HTTP-Post is done using JSON, Syntax see {RFC-7469 Section 3}\footnote{\url{https://tools.ietf.org/html/rfc7469\#section-3}}). There are WebServices like \url{https://report-uri.io/} out there which can be used to easily collect and visualize these reports.

\subsubsection{HPKP Client Support}
\label{subsubsection:HPKP Client Support}
HPKP is supported\footnote{\url{http://caniuse.com/\#feat=publickeypinning}} by these web browsers:
\begin{itemize*}
  \item Firefox version >= 35
	\item Chrome version >= 38
	\item Android Browser >= 44
	\item Opera version >= 25 
\end{itemize*}
Currently (16. Oct 2015) there is no HPKP Support in: Apple Safari, Microsoft Internet Explorer and Edge.

\subsubsection{HPKP Considerations}
\label{subsubsection:HPKP Considerations}
Before enabling HPKP it is recommended to consider following:
\begin{itemize*}
  \item Which Public-Keys to use for Pinning (Certificat + Backup-Certificate, CAs, Intermediate-CAs)
	\item Proper value of \emph{max-age}. Start testing with a short Period, increase Period after deployment.
	\item Be careful when using \emph{includeSubdomains}, are all your subdomains covered by the defined Public-Key-Hashes?
\end{itemize*}

The administrators are advised to watch for expiration of the SSL certificate and handle the renewal process with caution. If a SSL certificate is renewed without keeping the Public-Key (reusing the CSR) for an HPKP enabled domain name, the connection to site will break (without providing override mechanism to the end user).  

\subsubsection{Testing HPKP}
\label{subsubsection:Testing HPKP}
HPKP can be tested either using locally or through the Internet. 

There is a handy Bash-Script which uses OpenSSL for doing several SSL/TLS-Tests available at \url{https://testssl.sh/}
\begin{lstlisting}
wget -q https://testssl.sh/testssl.sh
wget -q https://testssl.sh/mapping-rfc.txt
chmod 755 ./testssl.sh
./testssl.sh https://your-domain.com

# Sample Output, just HSTS and HPKP Section (Full report is quite long!):
Strict Transport Security    182 days=15724800 s, includeSubDomains
Public Key Pinning # of keys: 2, 90 days=7776000 s, just this domain
           matching host key: pG3WsstDsfMkRdF3hBClXRKYxxKUJIOu8DwabG8MFrU
\end{lstlisting}

For local testing it is possible to utilize Google Chrome Web browser, just open the Chrome net-internals-URL: \url{chrome://net-internals/#hsts}.

For Mozilla Firefox there is an Plug-In provided by the "Secure Information Technology Center Austria" available: \url{https://demo.a-sit.at/firefox-plugin-highlighting-safety-information/}

Testing over the Internet can be conducted by Qualys SSL Labs test \url{https://www.ssllabs.com/ssltest/}. \emph{Public Key Pinning (HPKP)} information is located in the \emph{Protocol Details} section. 

There is also a fast online HPKP-only check at \url{https://report-uri.io/home/pkp_analyse}.

\subsubsection{References}
\begin{itemize*}
	\item OWASP: Certificate and Public Key Pinning: \url{https://www.owasp.org/index.php/Certificate_and_Public_Key_Pinning}
	\item HPKP Browser Compatibility List: \url{http://caniuse.com/\#feat=publickeypinning}
  \item RFC 7469:Public Key Pinning Extension for HTTP - Examples: \url{https://tools.ietf.org/html/rfc7469\#section-2.1.5}
\end{itemize*}


