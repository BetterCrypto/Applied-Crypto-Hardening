\section{Public Key Infrastructures}
\label{section:PKIs}

Public-Key Infrastructures try to solve the problem of verifying
whether a public key belongs to a given entity, so as to prevent Man
In The Middle attacks.

There are two approaches to achieve that: \emph{Certificate Authorities}
and the \emph{Web of Trust}.

Certificate Authorities (CAs) sign end-entities' certificates, thereby
associating some kind of identity (e.g. a domain name or an email
address) with a public key. CAs are used with TLS and S/MIME
certificates, and the CA system has a big list of possible and real
problems which are summarized in section \ref{sec:hardeningpki} and~\cite{https13}.

The Web of Trust is a decentralized system where people sign each
other's keys, so that there is a high chance that there is a ``trust
path'' from one key to another. This is used with PGP keys, and while
it avoids most of the problems of the CA system, it is more
cumbersome.

As alternatives to these public systems, there are two more choices:
running a private CA, and manually trusting keys (as it is used with
SSH keys or manually trusted keys in web browsers).

The first part of this section addresses how to obtain a certificate
in the CA system. The second part offers recommendations on how to
improve the security of your PKI.

\subsection{Certificate Authorities}
\label{sec:cas}
In order to get a certificate, you can find an external CA willing to issue
a certificate for you, run your own CA, or use self-signed certificates.
As always, there are advantages and disadvantages for every one of these
options; a balance of security versus usability needs to be found.

\subsubsection{Certificates From an External Certificate Authority}
\label{sec:signcertfromca}

There is a fairly large number of commercial CAs that will issue
certificates for money.  Some of the most ubiquitous commercial CAs are
Verisign, GoDaddy, and Teletrust.  However, there are also CAs that offer
certificates for free.  The most notable examples are StartSSL, which is a
company that offers some types of certificates for free, and CAcert, which
is a non-profit volunteer-based organization that does not charge at all
for issuing certificates.  Finally, in the research and education field, a
number of CAs exist that are generally well-known and well-accepted within
the higher-education community.

A large number of CAs is pre-installed in client software's or
operating system's``trust stores''; depending on your application, you
have to select your CA according to this, or have a mechanism to
distribute the chosen CA's root certificate to the clients.

When requesting a certificate from a CA, it is vital that you generate the
key pair yourself.  In particular, the private key should never be known to
the CA.  If a CA offers to generate the key pair for you, you should not
trust that CA.

Generating a key pair and a certificate request can be done with a number of
tools.  On Unix-like systems, it is likely that the OpenSSL suite is available
to you.  In this case, you can generate a private key and a corresponding
certificate request as follows:

\begin{lstlisting}
% openssl req -new -nodes -keyout <servername>.key -out <servername>.csr -newkey rsa:<keysize> -sha256
Country Name (2 letter code) [AU]:DE
State or Province Name (full name) [Some-State]:Bavaria
Locality Name (eg, city) []:Munich
Organization Name (eg, company) [Internet Widgits Pty Ltd]:Example
Organizational Unit Name (eg, section) []:Example Section
Common Name (e.g. server FQDN or YOUR name) []:example.com
Email Address []:admin@example.com

Please enter the following 'extra' attributes
to be sent with your certificate request
A challenge password []:
An optional company name []:
\end{lstlisting}

\subsubsection{Setting Up Your Own Certificate Authority}
\label{sec:setupownca}
In some situations it is advisable to run your own certificate authority.
Whether this is a good idea depends on the exact circumstances.  Generally
speaking, the more centralized the control of the systems in your
environment, the fewer pains you will have to go through to deploy your own
CA.  On the other hand, running your own CA maximizes the trust level that
you can achieve because it minimizes external trust dependencies.

Again using OpenSSL as an example, you can set up your own CA with the
following commands on a Debian system:

\begin{lstlisting}
% cd /usr/lib/ssl/misc
% sudo ./CA.pl -newca
\end{lstlisting}

Answer the questions according to your setup. Now that you have configured your basic settings and
issued a new root certificate, you can issue new certificates as follows:

\begin{lstlisting}
% cd /usr/lib/ssl/misc
% sudo ./CA.pl -newreq
\end{lstlisting}

Alternatively, software such as TinyCA~\cite{Wikipedia:TinyCA} that
acts as a ``wrapper'' around OpenSSL and tries to make life easier is
available.

\subsubsection{Creating a Self-Signed Certificate}
\label{sec:pki:selfsignedcert}
If the desired trust level is very high and the number of systems involved
is limited, the easiest way to set up a secure environment may be to use
self-signed certificates.  A self-signed certificate is not issued by any
CA at all, but is signed by the entity that it is issued to.  Thus, the
organizational overhead of running a CA is eliminated at the expense of
having to establish all trust relationships between entities manually.

With OpenSSL, you can self-sign a previously created certificate with this command:

\begin{lstlisting}
% openssl req -new -x509 -key privkey.pem -out cacert.pem -days 1095
\end{lstlisting}

You can also create a self-signed certificate in just one command:
\begin{lstlisting}
openssl req -new -x509 -keyout privkey.pem -out cacert.pem -days 1095 -nodes -newkey rsa:<keysize> -sha256
\end{lstlisting}

The resulting certificate will by default not be trusted by anyone at all,
so in order to be useful, the certificate will have to be made known a
priori to all parties that may encounter it.


\subsection{Hardening PKI}
\label{sec:hardeningpki}

In recent years several CAs were compromised by attackers in order to get a
hold of trusted certificates for malicious activities. In 2011 the Dutch CA
Diginotar was hacked and all certificates were revoked~\cite{diginotar-hack}.
Recently Google found certificates issued to them, which were not used by the
company~\cite{googlecahack}. The concept of PKIs heavily depends on the
security of CAs.  If they get compromised the whole PKI system will fail. Some
CAs tend to incorrectly issue certificates that were designated to do a
different job than what they were intended to by the CA~\cite{gocode}.

Therefore several security enhancements were introduced by different
organizations and vendors~\cite{tschofenig-webpki}. Currently two
methods are used, DANE~\cite{rfc6698} and Certificate
Pinning~\cite{draft-ietf-websec-key-pinning}. Google recently proposed
a new system to detect malicious CAs and certificates called Certificate
Transparency~\cite{certtransparency}. In addition, RFC 6844 describes Certification Authorization Records, a mechanism for domain name owners to signal which Certificate Authorities are authorized to issue certificates for their domain.

% \subsubsection{DANE}
% \label{sec:dane}

% \subsubsection{Certificate Pinning}
% \label{sec:certpinning}

\subsection{Certification Authorization Records}
\label{sec:caarecords}

RFC 6844 describes Certification Authorization Records, a mechanism for domain name owners to signal which Certificate Authorities are authorized to issue certificates for their domain.
 
When a CAA record is defined for a particular domain, it specifies that the domain owner requests Certificate Authorities to validate any request against the CAA record. If the certificate issuer is not listed in the CAA record, it should not issue the certificate.
 
The RFC also permits Certificate Evaluators to test an issued certificate against the CAA record, but should exercise caution, as the CAA record may change during the lifetime of a certificate, without affecting its validity.
 
CAA also supports an iodef property type which can be requested by a Certificate Authority to report certificate issue requests which are inconsistent with the issuer’s Certificate Policy.

\subsubsection{Configuration of CAA records}
\label{sec:pki:caarecords:configuration}
 
BIND supports CAA records as of version 9.9.6.
 
A CAA record can be configured by adding it to the zone file:

\begin{lstlisting}
$ORIGIN example.com
       CAA 0 issue "ca1.example"
       CAA 0 iodef "mailto:security@example.com"
\end{lstlisting}
 
If your organization uses multiple CA’s, you can configure multiple records:

\begin{lstlisting} 
      CAA 0 issue "ca1.example"
      CAA 0 issue "ca2.example"
\end{lstlisting}
 
“ca1.example” and “ca2.example” are unique identifiers for the CA you plan on using. These strings can be obtained from your Certificate Authority, and typically are its top level domain. An example is “letsencrypt.org” for the Let’s Encrypt CA operated by the Internet Security Research Group.
 
Knot-DNS supports CAA records as of version 2.2.0.
 
\subsubsection{Validation of CAA records}
\label{sec:pki:caarecords:validation}
 
Once a CAA record is deployed, it can be validated using the following dig query:
 
\begin{lstlisting} 
user@system:~$ dig CAA google.com
 
; <<>> DiG 9.10.3-P4-Debian <<>> CAA google.com
 
;; ANSWER SECTION:
google.com.          3600 IN   CAA  0 issue "symantec.com"
\end{lstlisting}

On older versions of Dig, which do not support CAA records, you can query the record type manually:

\begin{lstlisting} 
dig +short -t TYPE257 google.com
\# 19 0005697373756573796D616E7465632E636F6D
\end{lstlisting}

% This section deals with settings related to trusting CAs. However,
% our main recommendations for PKIs is: if you are able to run your
% own PKI and disable any other CA, do so. This makes sense most in
% environments where any machine-to-machine communication system
% compatibility with external entities is not an issue.
%% azet: this needs discussion! self-signed certificates simply do not
%% work in practices for real-world scenarios - i.e. websites that
%% actually serve a lot of people

% A good background on PKIs can be found in
% \footnote{\url{https://developer.mozilla.org/en/docs/Introduction_to_Public-Key_Cryptography}}
% \footnote{\url{http://cacr.uwaterloo.ca/hac/about/chap8.pdf}}
% \footnote{\url{https://www.verisign.com.au/repository/tutorial/cryptography/intro1.shtml}}
% .

% \todo{ts: Background and Configuration (EMET) of Certificate Pinning,
%   TLSA integration, When to use self-signed certificates, how to get
%   certificates from public CA authorities (CACert, StartSSL),
%   Best-practices how to create a CA and how to generate private
%   keys/CSRs, Discussion about OCSP and CRLs. TD: Useful Firefox
%   plugins: CipherFox, Conspiracy, Perspectives.}


% ``Certificate Policy''~\cite{Wikipedia:Certificate_Policy} (CA)
