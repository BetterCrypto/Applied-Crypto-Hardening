%hack.
\gdef\currentsectionname{DBs}
%%\subsection{Database Systems}
% This list is based on : https://en.wikipedia.org/wiki/Relational_database_management_system#Market_share

%% ---------------------------------------------------------------------- 
\subsection{Oracle}
\subsubsection{Tested with Versions}
\begin{itemize*}
\item We do not test this here, since we only reference other papers for Oracle so far.
\end{itemize*}


\subsubsection{References}
\begin{itemize*}
  \item Technical safety requirements by \emph{Deutsche Telekom AG} (German). Please read section 17.12 or pages 129 and following (Req 396 and Req 397) about SSL and ciphersuites \url{http://www.telekom.com/static/-/155996/7/technische-sicherheitsanforderungen-si}
\end{itemize*}



%% ---------------------------------------------------------------------- 
%%\subsection{SQL Server}
%%\todo{write this}



%% ---------------------------------------------------------------------- 
\subsection{MySQL}


\subsubsection{Tested with Versions}
\begin{itemize*}
  \item Debian Wheezy and MySQL 5.5
\end{itemize*}


\subsubsection{Settings}
\configfile{my.cnf}{31-31,104-109}{SSL configuration fo MySQL}


%\subsubsection{Additional settings}


%\subsubsection{Justification for special settings (if needed)}
% in case you have the need for further justifications why you chose this and that setting or if the settings do not fit into the standard Variant A or Variant B schema, please document this here


\subsubsection{References}
\begin{itemize*}
  \item MySQL Documentation on SSL Connections.\\\url{https://dev.mysql.com/doc/refman/5.5/en/ssl-connections.html}
\end{itemize*}


\subsubsection{How to test}
After restarting the server run the following query to see if the ssl settings are correct:
\begin{lstlisting}
show variables like '%ssl%';
\end{lstlisting}


%% ---------------------------------------------------------------------- 
\subsection{DB2}
\subsubsection{Tested with Version}
\begin{itemize*}
\item  We do not test this here, since we only reference other papers for DB2 so far.
\end{itemize*}


\subsubsection{Settings}
\paragraph{ssl\_cipherspecs:}
In the link above the whole SSL-configuration is described in-depth. The following command shows only how to set the recommended ciphersuites.
\begin{lstlisting}
# recommended and supported ciphersuites 

db2 update dbm cfg using SSL_CIPHERSPECS 
TLS_RSA_WITH_AES_256_CBC_SHA256,
TLS_RSA_WITH_AES_128_GCM_SHA256,
TLS_RSA_WITH_AES_128_CBC_SHA256,
TLS_ECDHE_RSA_WITH_AES_128_GCM_SHA256,
TLS_ECDHE_ECDSA_WITH_AES_128_GCM_SHA256,
TLS_ECDHE_RSA_WITH_AES_128_CBC_SHA256,
TLS_ECDHE_ECDSA_WITH_AES_128_CBC_SHA256,
TLS_RSA_WITH_AES_256_GCM_SHA384,
TLS_ECDHE_RSA_WITH_AES_256_GCM_SHA384,
TLS_ECDHE_ECDSA_WITH_AES_256_GCM_SHA384,
TLS_ECDHE_RSA_WITH_AES_256_CBC_SHA384,
TLS_ECDHE_ECDSA_WITH_AES_256_CBC_SHA384,
TLS_ECDHE_RSA_WITH_AES_256_CBC_SHA,
TLS_ECDHE_ECDSA_WITH_AES_256_CBC_SHA,
TLS_RSA_WITH_AES_256_CBC_SHA,
TLS_RSA_WITH_AES_128_CBC_SHA,
TLS_ECDHE_RSA_WITH_AES_128_CBC_SHA,
TLS_ECDHE_ECDSA_WITH_AES_128_CBC_SHA
\end{lstlisting}


\subsubsection{References}
\begin{itemize*}
  \item IBM Db2 Documentation on \emph{Supported cipher suites}.\\\url{http://pic.dhe.ibm.com/infocenter/db2luw/v9r7/index.jsp?topic=\%2Fcom.ibm.db2.luw.admin.sec.doc\%2Fdoc\%2Fc0053544.html}
\end{itemize*}

%% ---------------------------------------------------------------------- 

\subsection{PostgreSQL}
\subsubsection{Tested with Versions}
\begin{itemize*}
  \item Debian Wheezy and PostgreSQL 9.1
  \item Linux Mint 14 nadia / Ubuntu 12.10 quantal with PostgreSQL 9.1+136 and OpenSSL 1.0.1c
\end{itemize*}


\subsubsection{Settings}
\configfile{9.1/postgresql.conf}{80-81}{Enabling SSL in PostgreSQL}

To start in SSL mode the server.crt and server.key must exist in the server's data directory \$PGDATA.

Starting with version 9.2, you have the possibility to set the path manually.
\configfile{9.3/postgresql.conf}{85-87}{Certificate locations in PostgreSQL \(\geq\) 9.2}



\subsubsection{References}
\begin{itemize*}
  \item It's recommended to read ``Security and Authentication'' in the manual\footnote{\url{http://www.postgresql.org/docs/9.1/interactive/runtime-config-connection.html}}.
  \item PostgreSQL Documentation on \emph{Secure TCP/IP Connections with SSL}: \url{http://www.postgresql.org/docs/9.1/static/ssl-tcp.html}
  \item PostgreSQL Documentation on \emph{host-based authentication}: \url{http://www.postgresql.org/docs/current/static/auth-pg-hba-conf.html}
\end{itemize*}


\subsubsection{How to test}
To test your ssl settings, run psql with the sslmode parameter:
\begin{lstlisting}
psql "sslmode=require host=postgres-server dbname=database" your-username
\end{lstlisting}

