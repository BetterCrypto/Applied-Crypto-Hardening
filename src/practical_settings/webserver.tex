%%----------------------------------------------------------------------
\subsection{Apache}

Note that any cipher suite starting with EECDH can be omitted, if in doubt.
(Compared to the theory section, EECDH in Apache and ECDHE in OpenSSL are
synonyms~\footnote{https://www.mail-archive.com/openssl-dev@openssl.org/msg33405.html})

\subsubsection{Tested with Versions}
\begin{itemize*}
  \item Apache 2.2.22, Debian Wheezy with OpenSSL 1.0.1e
  \item Apache 2.4.6, Debian Jessie with OpenSSL 1.0.1e
  \item Apache 2.4.7, Ubuntu 14.04.2 Trusty with Openssl 1.0.1f
\end{itemize*}

\subsubsection{Settings}
Enabled modules \emph{SSL} and \emph{Headers} are required.

\configfile{default-ssl}{42-43,52-52,62-62,162-170}{SSL configuration for an Apache vhost}

\subsubsection{Additional settings}
You might want to redirect everything to \emph{https://} if possible. In Apache
you can do this with the following setting inside of a VirtualHost environment:

\configfile{hsts-vhost}{}{https auto-redirect vhost}

%\subsubsection{Justification for special settings (if needed)}

\subsubsection{References}
\begin{itemize*}
  \item Apache2 Docs on SSL and TLS: \url{https://httpd.apache.org/docs/2.4/ssl/}
\end{itemize*}


\subsubsection{How to test}
See appendix \ref{cha:tools}


%%----------------------------------------------------------------------
\subsection{lighttpd}

\subsubsection{Tested with Versions}
\begin{itemize*}
  \item lighttpd/1.4.31-4 with OpenSSL 1.0.1e on Debian Wheezy
  \item lighttpd/1.4.33 with OpenSSL 0.9.8o on Debian Squeeze (note that TLSv1.2 does not work in openssl 0.9.8 thus not all ciphers actually work)
  \item lighttpd/1.4.28-2 with OpenSSL 0.9.8o on Debian Squeeze (note that TLSv1.2 does not work in openssl 0.9.8 thus not all ciphers actually work)
  \item lighttpd/1.4.31, Ubuntu 14.04.2 Trusty with Openssl 1.0.1f
\end{itemize*}


\subsubsection{Settings}
\configfile{10-ssl.conf}{3-15}{SSL configuration for lighttpd}

Starting with lighttpd version 1.4.29 Diffie-Hellman and Elliptic-Curve Diffie-Hellman key agreement protocols are supported.
By default, elliptic curve "prime256v1" (also "secp256r1") will be used, if no other is given.
To select special curves, it is possible to set them using the configuration options \verb|ssl.dh-file| and \verb|ssl.ec-curve|.

\configfile{10-ssl-dh.conf}{11-13}{SSL EC/DH configuration for lighttpd}

Please read section \ref{section:DH} for more information on Diffie Hellman key exchange and elliptic curves.

\subsubsection{Additional settings}
As for any other webserver, you might want to automatically redirect \emph{http://}
traffic toward \emph{https://}. It is also recommended to set the environment variable
\emph{HTTPS}, so the PHP applications run by the webserver can easily detect
that HTTPS is in use.

\configfile{11-hsts.conf}{}{https auto-redirect configuration}

\subsubsection{Additional information}
The config option \emph{honor-cipher-order} is available since 1.4.30, the
supported ciphers depend on the used OpenSSL-version (at runtime). ECDHE has to
be available in OpenSSL at compile-time, which should be default. SSL
compression should by deactivated by default at compile-time (if not, it's
active).

Support for other SSL-libraries like GnuTLS will be available in the upcoming
2.x branch, which is currently under development.


\subsubsection{References}
\begin{itemize*}
  \item HTTPS redirection: \url{http://redmine.lighttpd.net/projects/1/wiki/HowToRedirectHttpToHttps}
  \item Lighttpd Docs SSL: \url{http://redmine.lighttpd.net/projects/lighttpd/wiki/Docs\_SSL}
  \item Release 1.4.30 (How to mitigate BEAST attack) \url{http://redmine.lighttpd.net/projects/lighttpd/wiki/Release-1\_4\_30}
  \item SSL Compression disabled by default: \url{http://redmine.lighttpd.net/issues/2445}
\end{itemize*}


\subsubsection{How to test}
See appendix \ref{cha:tools}


%%----------------------------------------------------------------------
\subsection{nginx}

\subsubsection{Tested with Version}
\begin{itemize*}
  \item 1.4.4 with OpenSSL 1.0.1e on OS X Server 10.8.5
  \item 1.2.1-2.2+wheezy2 with OpenSSL 1.0.1e on Debian Wheezy
  \item 1.4.4 with OpenSSL 1.0.1e on Debian Wheezy
  \item 1.2.1-2.2~bpo60+2 with OpenSSL 0.9.8o on Debian Squeeze (note that TLSv1.2 does not work in openssl 0.9.8 thus not all ciphers actually work)
  \item 1.4.6 with OpenSSL 1.0.1f on Ubuntu 14.04.2 LTS
\end{itemize*}


\subsubsection{Settings}
\configfile{default}{113-118}{SSL settings for nginx}
If you absolutely want to specify your own DH parameters, you can specify them via

\begin{lstlisting}
ssl_dhparam file;
\end{lstlisting}

However, we advise you to read section \ref{section:DH} and stay with the standard IKE/IETF parameters (as long as they are \textgreater 1024 bits).

\subsubsection{Additional settings}
If you decide to trust NIST's ECC curve recommendation, you can add the following line to nginx's configuration file to select special curves:

\configfile{default-ec}{119-119}{SSL EC/DH settings for nginx}

You might want to redirect everything to \emph{https://} if possible. In Nginx you can do this with the following setting:

\configfile{default-hsts}{29-29}{https auto-redirect in nginx}

The variable \emph{\$server\_name} refers to the first \emph{server\_name} entry in your config file. If you specify more than one \emph{server\_name} only the first will be taken. Please be sure to not use the \emph{\$host} variable here because it contains data controlled by the user.

\subsubsection{References}
\begin{itemize*}
  \item \url{http://nginx.org/en/docs/http/ngx_http_ssl_module.html}
  \item \url{http://wiki.nginx.org/HttpSslModule}
\end{itemize*}

\subsubsection{How to test}
See appendix \ref{cha:tools}


%%----------------------------------------------------------------------
\subsection{Cherokee}

\subsubsection{Tested with Version}
\begin{itemize*}
    \item Cherokee/1.2.104 on Debian Wheezy with OpenSSL 1.0.1e 11 Feb 2013
\end{itemize*}

\subsubsection{Settings}

The configuration of the cherokee webserver is performed by an admin interface available via the web. It then writes the configuration to \texttt{/etc/cherokee/cherokee.conf}, the important lines of such a configuration file can be found at the end of this section.

\begin{itemize*}
    \item General Settings
    \begin{itemize*}
        \item Network
        \begin{itemize*}
            \item \emph{SSL/TLS back-end}: \emph{OpenSSL/libssl}
        \end{itemize*}
        \item Ports to listen
        \begin{itemize*}
            \item Port: 443, TLS: TLS/SSL port
        \end{itemize*}
    \end{itemize*}
    \item Virtual Servers, For each vServer on tab \emph{Security}:
    \begin{itemize*}
        \item \emph{Required SSL/TLS Values}: Fill in the correct paths for \emph{Certificate} and \emph{Certificate key}
        \item Advanced Options
        \begin{itemize*}
            \item \emph{Ciphers}: \texttt{EDH+CAMELLIA:EDH+aRSA:EECDH+aRSA+AESGCM:EECDH+aRSA+SHA384:\newline EECDH+aRSA+SHA256:EECDH:+CAMELLIA256:+AES256:+CAMELLIA128:+AES128:\newline+SSLv3:!aNULL!eNULL:!LOW:!3DES:!MD5:!EXP:!PSK:!DSS:!RC4:!SEED:\newline!ECDSA:CAMELLIA256-SHA:AES256-SHA:CAMELLIA128-SHA:AES128-SHA}
            \item \emph{Server Preference}: Prefer
            \item \emph{Compression}: Disabled
        \end{itemize*}
    \end{itemize*}
    \item Advanced: TLS
    \begin{itemize*}
        \item SSL version 2 and SSL version 3: No
        \item TLS version 1, TLS version 1.1 and TLS version 1.2: Yes
    \end{itemize*}
\end{itemize*}

\subsubsection{Additional settings}
For each vServer on the Security tab it is possilbe to set the Diffie Hellman length to up to 4096 bits. We recommend to use \textgreater 1024 bits.
More information about Diffie-Hellman and which curves are recommended can be found in section \ref{section:DH}.

In Advanced: TLS it is possible to set the path to a Diffie Hellman parameters file for 512, 1024, 2048 and 4096 bits.

HSTS can be configured on host-basis in section \emph{vServers} / \emph{Security} / \emph{HTTP Strict Transport Security (HSTS)}:
\begin{itemize*}
    \item \emph{Enable HSTS}: Accept
    \item \emph{HSTS Max-Age}: 15768000
    \item \emph{Include Subdomains}: depends on your setup
\end{itemize*}

To redirect HTTP to HTTPS, configure a new rule per Virtual Server in the \emph{Behavior} tab. The rule is \emph{SSL/TLS} combined with a \emph{NOT} operator. As \emph{Handler} define \emph{Redirection} and use \texttt{/(.*)\$} as \emph{Regular Expression} and \emph{https://\$\{host\}/\$1} as \emph{Substitution}.

\configfile{cherokee.conf}{3-4,12-12,17-19,26-32,52-57}{SSL configuration for cherokee}

\subsubsection{References}
\begin{itemize*}
  \item Cookbook: SSL, TLS and certificates: \url{http://cherokee-project.com/doc/cookbook_ssl.html}
  \item Cookbook: Redirecting all traffic from HTTP to HTTPS: \url{http://cherokee-project.com/doc/cookbook_http_to_https.html}
\end{itemize*}


\subsubsection{How to test}
See appendix \ref{cha:tools}


%%----------------------------------------------------------------------
\subsection{MS IIS}
\label{sec:ms-iis}

To configure SSL/TLS on Windows Server IIS Crypto can be used.~\footnote{\url{https://www.nartac.com/Products/IISCrypto/}}
Simply start the Programm, no installation required. The tool changes the registry keys described below.
A restart is required for the changes to take effect.

\begin{figure}[p]
  \centering
  \includegraphics[width=0.411\textwidth]{img/IISCryptoConfig.png}
  \caption{IIS Crypto Tool}
  \label{fig:IISCryptoConfig}
\end{figure}

Instead of using the IIS Crypto Tool the configuration can be set
using the Windows Registry. The following Registry keys apply to the
newer Versions of Windows (Windows 7, Windows Server 2008, Windows
Server 2008 R2, Windows Server 2012 and Windows Server 2012 R2). For detailed
information about the older versions see the Microsoft knowledgebase
article. \footnote{\url{http://support.microsoft.com/kb/245030/en-us}}
\begin{lstlisting}[breaklines]
  [HKEY_LOCAL_MACHINE\SYSTEM\CurrentControlSet\Control\SecurityProviders\Schannel]
  [HKEY_LOCAL_MACHINE\SYSTEM\CurrentControlSet\Control\SecurityProviders\Schannel\Ciphers]
  [HKEY_LOCAL_MACHINE\SYSTEM\CurrentControlSet\Control\SecurityProviders\Schannel\CipherSuites]
  [HKEY_LOCAL_MACHINE\SYSTEM\CurrentControlSet\Control\SecurityProviders\Schannel\Hashes]
  [HKEY_LOCAL_MACHINE\SYSTEM\CurrentControlSet\Control\SecurityProviders\Schannel\KeyExchangeAlgorithms]
  [HKEY_LOCAL_MACHINE\SYSTEM\CurrentControlSet\Control\SecurityProviders\Schannel\Protocols]
\end{lstlisting}

\subsubsection{Tested with Version}
\begin{itemize*}
  \item Windows Server 2008
  \item Windows Server 2008 R2
  \item Windows Server 2012
  \item Windows Server 2012 R2
\end{itemize*}

\begin{itemize*}
  \item Windows Vista and Internet Explorer 7 and upwards
  \item Windows 7 and Internet Explorer 8 and upwards
  \item Windows 8 and Internet Explorer 10 and upwards
  \item Windows 8.1 and Internet Explorer 11
\end{itemize*}






\subsubsection{Settings}
When trying to avoid RC4 (RC4 biases) as well as CBC (BEAST-Attack) by using GCM and to support perfect
forward secrecy, Microsoft SChannel (SSL/TLS, Auth,.. Stack) supports
ECDSA but lacks support for RSA signatures (see ECC suite
B doubts\footnote{\url{http://safecurves.cr.yp.to/rigid.html}}).

Since one is stuck with ECDSA, an elliptic curve certificate needs to be used.

The configuration of cipher suites MS IIS will use, can be configured in one
of the following ways:
\begin{enumerate}
  \item Group Policy \footnote{\url{http://msdn.microsoft.com/en-us/library/windows/desktop/bb870930(v=vs.85).aspx}}
  \item Registry  \footnote{\url{http://support.microsoft.com/kb/245030 }}
  \item IIS Crypto~\footnote{\url{https://www.nartac.com/Products/IISCrypto/}}
  \item Powershell
\end{enumerate}


Table~\ref{tab:MS_IIS_Client_Support} shows the process of turning on
one algorithm after another and the effect on the supported clients
tested using https://www.ssllabs.com.

\verb|SSL 3.0|, \verb|SSL 2.0| and \verb|MD5| are turned off.
\verb|TLS 1.0| and \verb|TLS 1.2| are turned on.

\ctable[%
caption={Client support},
label=tab:MS_IIS_Client_Support,
]{ll}{}{%
\FL    Cipher Suite & Client
\ML    \lstinline+TLS_ECDHE_ECDSA_WITH_AES_128_GCM_SHA256+ & only IE 10,11, OpenSSL 1.0.1e
\NN    \lstinline+TLS_ECDHE_ECDSA_WITH_AES_128_CBC_SHA256+ & Chrome 30, Opera 17, Safari 6+
\NN    \lstinline+TLS_ECDHE_ECDSA_WITH_AES_128_CBC_SHA+ & FF 10-24, IE 8+, Safari 5, Java 7
\LL}

Table~\ref{tab:MS_IIS_Client_Support} shows the algorithms from
strongest to weakest and why they need to be added in this order. For
example insisting on SHA-2 algorithms (only first two lines) would
eliminate all versions of Firefox, so the last line is needed to
support this browser, but should be placed at the bottom, so capable
browsers will choose the stronger SHA-2 algorithms.

\verb|TLS_RSA_WITH_RC4_128_SHA| or equivalent should also be added if
MS Terminal Server Connection is used (make sure to use this only in a
trusted environment). This suite will not be used for SSL, since we do
not use a RSA Key.

% \verb|TLS_ECDHE_ECDSA_WITH_AES_128_GCM_SHA256| ... only supported by: IE 10,11, OpenSSL 1.0.1e
% \verb|TLS_ECDHE_ECDSA_WITH_AES_128_CBC_SHA256| ... Chrome 30, Opera 17, Safari 6+
% \verb|TLS_ECDHE_ECDSA_WITH_AES_128_CBC_SHA| ... Firefox 10-24, IE 8+, Safari 5, Java 7

Clients not supported:
\begin{enumerate}
  \item Java 6
  \item WinXP
  \item Bing
\end{enumerate}


\subsubsection{Additional settings}
%Here you can add additional settings
It's recommended to use Strict-Transport-Security: max-age=15768000
for detailed information visit the
\footnote{\url{http://www.iis.net/configreference/system.webserver/httpprotocol/customheaders}}
Microsoft knowledgebase.

You might want to redirect everything to http\textbf{s}:// if possible. In IIS you can do this with the following setting by Powershell:

\begin{lstlisting}[breaklines]
Set-WebConfiguration -Location "$WebSiteName/$WebApplicationName" `
    -Filter 'system.webserver/security/access' `
    -Value "SslRequireCert"
\end{lstlisting}

\subsubsection{Justification for special settings (if needed)}
% in case you have the need for further justifications why you chose this and that setting or if the settings do not fit into the standard Variant A or Variant B schema, please document this here


\subsubsection{References}
\begin{itemize*}
\item \url{http://support.microsoft.com/kb/245030/en-us}
\item \url{http://support.microsoft.com/kb/187498/en-us}
\end{itemize*}

% add any further references or best practice documents here


\subsubsection{How to test}
See appendix \ref{cha:tools}

\FloatBarrier % the preceding section has several figures. Floating them too far away might get confusing for readers.
%%----------------------------------------------------------------------

%%% Local Variables:
%%% mode: latex
%%% TeX-master: "../applied-crypto-hardening"
%%% End:
