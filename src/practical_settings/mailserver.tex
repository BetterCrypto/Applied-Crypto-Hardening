% hack.
\gdef\currentsectionname{MailServers}
This section documents the most common mail (SMTP) and IMAPs/POPs servers. Another option to secure IMAPs/POPs servers is to place them behind an stunnel server.


%% ----------------------------------------------------------------------
\subsection{SMTP in general}
\label{subsection:smtp_general}
SMTP usually makes use of opportunistic TLS. This means that an MTA will accept TLS connections when asked for it during handshake but will not require it. One should always support incoming opportunistic TLS and always try TLS handshake outgoing.

Furthermore a mailserver can operate in three modes:
\begin{itemize*}
  \item As MSA (Mail Submission Agent) your mailserver receives mail from your clients MUAs (Mail User Agent).
  \item As receiving MTA (Mail Transmission Agent, MX)
  \item As sending MTA (SMTP client)
\end{itemize*}
We recommend the following basic setup for all modes:
\begin{itemize*}
  \item correctly setup MX, A and PTR RRs without using CNAMEs at all.
  \item enable encryption (opportunistic TLS)
  \item do not use self signed certificates
\end{itemize*}

For SMTP client mode we additionally recommend:
\begin{itemize*}
  \item the hostname used as HELO must match the PTR RR
  \item setup a client certificate (most server certificates are client certificates as well)
  \item either the common name or at least an alternate subject name of your certificate must match the PTR RR
  \item do not modify the cipher suite for client mode
\end{itemize*}

For MSA operation we recommend:
\begin{itemize*}
  \item listen on submission port 587
  \item enforce SMTP AUTH even for local networks
  \item do not allow SMTP AUTH on unencrypted connections
  \item optionally use the recommended cipher suites if (and only if) all your connecting MUAs support them
\end{itemize*}


% Note that (with the exception of MSA mode), it might be better to allow any cipher suite -- since any encryption is better than no encryption when it comes to opportunistic TLS.

We strongly recommend to allow all cipher suites for anything but MSA
mode, because the alternative is plain text transmission.

%% ----------------------------------------------------------------------
\subsection{Dovecot}


\subsubsection{Tested with Version}
\begin{itemize*}
  \item Dovecot 2.1.7, Debian Wheezy (without ``ssl\_prefer\_server\_ciphers'' setting)
  \item Dovecot 2.2.9, Debian Jessie
  \item 2.0.19apple1 on OS X Server 10.8.5 (without ``ssl\_prefer\_server\_ciphers'' setting)
  \item Dovecot 2.2.9 on Ubuntu 14.04 trusty
\end{itemize*}

\subsubsection{Settings}
% Example: http://dovecot.org/list/dovecot/2013-October/092999.html

\configfile{10-ssl.conf}{48-55}{Dovecot SSL configuration}

\subsubsection{Additional info}
Dovecot 2.0, 2.1: Almost as good as dovecot 2.2. Dovecot does not ignore unknown configuration parameters. Does not support
ssl\_prefer\_server\_ciphers

\subsubsection{Limitations}
Dovecot currently does not support disabling TLS compression. Furthermore, DH
parameters greater than 1024bit are not supported. The most recent version
2.2.7 of Dovecot implements configurable DH parameter length
\footnote{\url{http://hg.dovecot.org/dovecot-2.2/rev/43ab5abeb8f0}}.

%\subsubsection{Justification for special settings (if needed)}

% in case you have the need for further justifications why you chose this and that setting or if the settings do not fit into the standard Variant A or Variant B schema, please document this here

\subsubsection{References}
\begin{itemize*}
  \item \url{http://wiki2.dovecot.org/SSL}
\end{itemize*}

% add any further references or best practice documents here

\subsubsection{How to test}
% describe here or point the admin to tools (can be a simple footnote or \ref{} to  the tools section) which help the admin to test his settings.
\begin{lstlisting}
openssl s_client -crlf -connect SERVER.TLD:993
\end{lstlisting}


%% ----------------------------------------------------------------------
\subsection{cyrus-imapd}
\subsubsection{Tested with Versions}
\begin{itemize*}
  \item 2.4.17
\end{itemize*}

\subsubsection{Settings}

To activate SSL/TLS configure your certificate with
\configfile{imapd.conf}{206-206,209-209}{Activating TLS in cyrus}

Do not forget to add necessary intermediate certificates to the .pem file.

Limiting the ciphers provided may force (especially older) clients to connect without encryption at all! Sticking to the defaults is recommended.

If you still want to force strong encryption use
\configfile{imapd.conf}{263-263}{TLS cipher selection in cyrus}

cyrus-imapd loads hardcoded 1024 bit DH parameters using get\_rfc2409\_prime\_1024() by default. If you want to load your own DH parameters add them PEM encoded to the certificate file given in tls\_cert\_file. Do not forget to re-add them after updating your certificate.

To prevent unencrypted connections on the STARTTLS ports you can set
\configfile{imapd.conf}{131-131}{Force encrypted connections in cyrus}
This way MUAs can only authenticate with plain text authentication schemes after issuing the STARTTLS command. Providing CRAM-MD5 or DIGEST-MD5 methods is not recommended.

To support POP3/IMAP on ports 110/143 with STARTTLS and POP3S/IMAPS on ports
995/993 check the SERVICES section in \texttt{cyrus.conf}
\configfile{cyrus.conf}{28-28,31-34,71-71}{STARTTLS for POP3/IMAP and POP3S/IMAPS in cyrus}


\subsubsection{Limitations}
cyrus-imapd currently (2.4.17, trunk) does not support elliptic curve cryptography. Hence, ECDHE will not work even if defined in your cipher list.

Currently there is no way to prefer server ciphers or to disable compression.

There is a working patch for all three features:
\url{https://bugzilla.cyrusimap.org/show_bug.cgi?id=3823}

\subsubsection{How to test}
\begin{lstlisting}
openssl s_client -crlf -connect SERVER.TLD:993
\end{lstlisting}

% XXX config von Adi?
% sslVersion = TLSv1
% ciphers = EDH+CAMELLIA256:EDH+aRSA:+SSLv3:!aNULL:!eNULL:!LOW:!3DES:!MD5:!EXP:!PSK:!SRP:!DSS:!RC4:!SEED:-AES128:!CAMELLIA128:!ECDSA:AES256-SHA:EDH+AES128;
% options = CIPHER_SERVER_PREFERENCE
% TIMEOUTclose = 1


%% ----------------------------------------------------------------------
\subsection{Postfix}

\subsubsection{Tested with Versions}
\begin{itemize*}
  \item Postfix 2.9.6, Debian Wheezy
\end{itemize*}

TLS in Postfix may take place in two Postfix programs - the \verb|smtpd| SMTP server and the \verb|smtp| SMTP client. This compartmentalized architecture is reflected in the following subsections. We'll start with the \verb|smtpd| SMTP server first.

\subsubsection{TLS in Postfix' smtpd SMTP Server}
As discussed in section \ref{subsection:smtp_general} "any encrypted mail transport is better than forcing a remote client to resort to plaintext transport" simply because the server offered a limited cipher suite and the client couldn't find any cipher to use. Thus, unless you setup a dedicated and controlled SMTP server (e.g. on submission port \verb|587|), you should not restrict the list of supported ciphers for a publicly referenced MX host listening on the default Port \verb|25|.

In order to enable opportunistic TLS for a publicly referenced SMTP server add the following configuration to Postfix' \verb|main.cf|:

\configfile{main.cf}{20-35}{Opportunistic TLS in Postfix}

%% I (cm) consider the generation of own DH parameters to be voodoo until
%% someone can explain the contrary. They are, after all, public, and
%% I found no research that would show that long-term use of a
%% parameter set would weaken the DH exchange. Also notice that IPSEC
%% uses fixed parameter sets only.
%&
%% also notice the following comment from  src/tls/tls_dh.c:
%% * Compiled-in EDH primes (the compiled-in generator is always 2). These are
%% * used when no parameters are explicitly loaded from a site-specific file.
%% *
%% * 512-bit parameters are used for export ciphers, and 1024-bit parameters are
%% * used for non-export ciphers. An ~80-bit strong EDH key exchange is really
%% * too weak to protect 128+ bit keys, but larger DH primes are
%% * computationally expensive. When greater security is required, use EECDH.

%% First, you need to generate Diffie Hellman parameters (please first take a look at the section \ref{section:RNGs}):

%% \todo{FIXME: this is a really weak setting! See also: http://postfix.1071664.n5.nabble.com/postfix-hardening-what-can-we-do-td61874.html}
%% \begin{lstlisting}
%%   % openssl gendh -out /etc/postfix/dh_param_512.pem -2 512
%%   % openssl gendh -out /etc/postfix/dh_param_1024.pem -2 1024
%% \end{lstlisting}

%% Next, we specify these DH parameters in \verb|main.cf|:

%% \begin{lstlisting}
%%   smtpd_tls_dh512_param_file = /etc/postfix/dh_param_512.pem
%%   smtpd_tls_dh1024_param_file = /etc/postfix/dh_param_1024.pem
%% \end{lstlisting}

\paragraph{MSA:}
Things change if you plan to run a dedicated \verb|smtpd| SMTP server listening on the submission port \verb|587|. This port is special because MUAs should use it to submit mail. In order to do so they MUST authenticate (SMTP AUTH).

The most widely supported AUTH mechanisms in MUAs are PLAIN and LOGIN. These mechanisms allow AUTH passwords to be stored in crypted backends, but they are transmitted base64 encoded only and SHOULD be protected by an encrypted transport layer. Since the clients are more or less well-known you may limit the cipher suite in this controlled environment.
 
The following MSA configuration demands TLS (mandatory TLS) from all clients connecting and limits the cipher suite at the same time. The configuration needs to be added in two Postfix configuration files:

The first settings in Postfix \verb|main.cf| establish new configuration parameters and assigns TLS options as desired. All configuration names begin with \verb|mua_| which are non-default names on purpose. They create a namespace that cannot collide with Postfix own \verb|smtpd_| parameter names adressing the \verb|smtpd| SMTP server -- they will never accidentally influence TLS behaviour of other SMTP servers, e.g. the one listening on port \verb|25|.

\configfile{main.cf}{37-45}{MSA TLS configuration in Postfix}

The second file that needs to be edited is Postfix \verb|master.cf|. The changes enable the Postfix submission service, create options for mandatory TLS and server-side cipher control and finally map settings from \verb|main.cf| onto the newly created submission server.

\configfile{master.cf}{12-18}{MSA smtpd service configuration in Postfix}

%% commented until PFS has been described
%% For those users who want to use EECDH key exchange, it is possible to customize this via:
%% \configfile{main.cf}{45-45}{EECDH customization in Postfix}
%% The default value since Postfix 2.8 is ``strong''.

%% \subsubsection{Settings for Opportunistic TLS using DANE}
%% 
%% If the DNS-domain the Postfix smtp client connects to is DNSSEC-enabled and if that domain offers a TLSA ressource record 
%% 
%% As of version 2.11 Postfix may also automatically identify 
%% 
%% supports DNS-Based Authentication of Named Entities (DANE) via DNSSEC-enabled domains.

\subsubsection{TLS in Postfix' smtp SMTP Client}

Enabling opportunistic TLS in the Postfix \verb|smtp| client is simple. Set \verb|smtp_tls_security_level| to \verb|may| and you are done. Additionally, if you want to and if your certificate has been signed for client usage also and not only server usage, provide the paths to your x509 certificate and key. When a server asks your Postfix \verb|smtp| client to identify itself with a TLS client certificate Postfix will send what you have provided during configuration.

\textbf{Note:} Asking a client for its TLS certificate is not default behaviour. It has to be enabled on the SMTP server side explicitly. It often remains disabled because not all client are able to handle TLC client certificate requests properly. Postfix has not been reported to be among these erroneous clients.

%% Mist klappt nicht \ref{section:DNS}

\configfile{main.cf}{47-58}{TLS client configuration in Postfix}

In the example above \verb|smtp_tls_note_starttls_offer| causes Postfix to log servers that offered \verb|STARTTLS|. This offers a convenient way to find out about destinations to whom explicit SMTP TLS policies could be written down in \verb|smtp_tls_policy_maps| (see also: \verb|postconf(5)|).

Setting up policies in \verb|smtp_tls_policy_maps| is a first step towards identified TLS. Identified TLS requires to verify the servers identity and putting it down in \verb|smtp_tls_policy_maps| so Postfix' \verb|smtp| client is able to use the provided data to identify the server it communicates with.

Usually this means to extract the servers' x509 certificate fingerprint, call the team that runs the server, verify the fingerprint and - once verified - write it down so Postfix can use it. This is costly and error-prone and only few people make that extra effort.

\paragraph{DANE} DNS-Based Authentication of Named Entities (DANE) automates this task. It uses a DNSSEC-enabled domain to distribute the SMTP servers certificate via TLSA Ressource Records (see: ???). Postfix (2.11.1) is the world's first MTA to implement "SMTP security via opportunistic DANE TLS". Using TLSA RRs, distributed via a DNSSEC-enabled domain, it can automatically identify a remote server offering \verb|STARTTLS|. This adds another level of security, since it eliminates Man-in-the-Middle-attacks.

\textbf{Note:} TLSA RRs allow to publish publicly signed certificates from CAs, but also self-signed ones. If you don't trust in CAs DNSSEC-enabled distribution of server identities is a good way to go.

Enabling DANE TLS in Postfix' \verb|smtp| client is simple. It requires a DNSSEC-aware DNS Resolver as described in \ref{subsection:Resolver} and the following options in \verb|main.cf|:

\configfile{main.cf}{61-65}{DANE TLS client configuration in Postfix}

Once you've reloaded Postfix you will begin to notice logo lines indicating identified TLS like this:
\begin{lstlisting}
Jul 11 12:20:01 mail postfix/smtp[9067]: Verified TLS connection established to mail.sys4.de[194.126.158.139]:25: TLSv1 with cipher ECDHE-RSA-AES256-SHA (256/256 bits)
\end{lstlisting}

This testifies Postfix was able to identify the remote server via DNS-Based Authentication of Named Entities.

If you want to offer DANE to other mail servers you need to DNSSEC-enable your MX domain. This is described in \ref{chapter:DNS}.


\subsubsection{Limitations}
tls\_ssl\_options is supported from Postfix 2.11 onwards. You can
leave the statement in the configuration for older versions, it will
be ignored.

tls\_preempt\_cipherlist is supported from Postfix 2.8 onwards. Again,
you can leave the statement in for older versions.

\subsubsection{References}
Refer to \url{http://www.postfix.org/TLS_README.html} for an in-depth
discussion.

\subsubsection{Additional settings}
Postfix has two sets of built-in DH parameters that can be overridden
with the \verb|smtpd_tls_dh512_param_file|
and \verb|smtpd_tls_dh1024_param_file| options. The ``dh512''
parameters are used for export ciphers, while the ``dh1024'' ones are
used for all other ciphers.

The ``bit length'' in those parameter names is just a name, so one
could use stronger parameter sets; it should be possible to e.g. use the
IKE Group14 parameters (see section \ref{section:DH}) without much
interoperability risk, but we have not tested this yet.

% \subsubsection{Justification for special settings (if needed)}
% no special settings


\subsubsection{How to test}
You can check the effect of the settings with the following command:
\begin{lstlisting}
$ zegrep "TLS connection established from.*with cipher" /var/log/mail.log | awk '{printf("%s %s %s %s\n", $12, $13, $14, $15)}' | sort | uniq -c | sort -n
      1 SSLv3 with cipher DHE-RSA-AES256-SHA
     23 TLSv1.2 with cipher DHE-RSA-AES256-GCM-SHA384
     60 TLSv1 with cipher ECDHE-RSA-AES256-SHA
    270 TLSv1.2 with cipher ECDHE-RSA-AES256-GCM-SHA384
    335 TLSv1 with cipher DHE-RSA-AES256-SHA
\end{lstlisting}

\begin{lstlisting}
openssl s_client -starttls smtp -crlf -connect SERVER.TLD:25
\end{lstlisting}

%% ----------------------------------------------------------------------

\subsection{Exim}

\subsubsection{Tested with Versions}
\begin{itemize*}
  \item Exim 4.82, Debian Jessie
\end{itemize*}


It is highly recommended to read
\url{http://exim.org/exim-html-current/doc/html/spec_html/ch-encrypted_smtp_connections_using_tlsssl.html}
first.

\paragraph{MSA mode (submission):}
In the main config section of Exim add:
\configfile{configure.msa}{153-154}{Certificate selection in Exim (MSA)}
Don't forget to add intermediate certificates to the .pem file if needed.

Tell Exim to advertise STARTTLS in the EHLO answer to everyone:
\configfile{configure.msa}{145-145}{TLS advertise in Exim (MSA)}

If you want to support legacy SMTPS on port 465, and STARTTLS on smtp(25)/submission(587) ports set
\configfile{configure.msa}{165-166}{STARTTLS and SMTPS in Exim (MSA)}

It is highly recommended to limit SMTP AUTH to SSL connections only. To do so add
\configfile{configure.msa}{813-813}{SSL-only authentication in Exim (MSA)}
to every authenticator defined.

Add the following rules on top of your acl\_smtp\_mail:
\configfile{configure.msa}{111-111,501-505}{Submission mode in Exim (MSA)}
This switches Exim to submission mode and allows addition of missing ``Message-ID'' and ``Date'' headers.

It is not advisable to restrict the default cipher list for MSA mode if you don't know all connecting MUAs. If you still want to define one please consult the Exim documentation or ask on the exim-users mailinglist.
% Exim maintainers do not recommend to change default ciphers
% I think we shouldn't, too
%use:
%\begin{lstlisting}
%  tls_require_ciphers = <...recommended ciphersuite...>
%\end{lstlisting}

The cipher used is written to the logfiles by default. You may want to add
\begin{lstlisting}
log_selector = <whatever your log_selector already contains> +tls_certificate_verified +tls_peerdn +tls_sni
\end{lstlisting}
to get even more TLS information logged.


\paragraph{Server mode (incoming):}
In the main config section of Exim add:
\configfile{configure.server}{152-153}{Certificate selection in Exim (Server)}
don't forget to add intermediate certificates to the .pem file if needed.

Tell Exim to advertise STARTTLS in the EHLO answer to everyone:
\configfile{configure.server}{144-144}{TLS advertise in Exim (Server)}

Listen on smtp(25) port only
\configfile{configure.server}{166-166}{STARTTLS on SMTP in Exim (Server)}

It is not advisable to restrict the default cipher list for opportunistic encryption as used by SMTP. Do not use cipher lists recommended for HTTPS! If you still want to define one please consult the Exim documentation or ask on the exim-users mailinglist.
% Exim maintainers do not recommend to change default ciphers
% We shouldn't, too
%use:
%\begin{lstlisting}
%  tls_require_ciphers = <...recommended ciphersuite...>
%\end{lstlisting}

If you want to request and verify client certificates from sending hosts set
\configfile{configure.server}{154-155}{TLS certificate verifiaction in Exim (Server)}

tls\_try\_verify\_hosts only reports the result to your logfile. If you want to disconnect such clients you have to use
\begin{lstlisting}
tls_verify_hosts = *
\end{lstlisting}

The cipher used is written to the logfiles by default. You may want to add
\begin{lstlisting}
log_selector = <whatever your log_selector already contains> +tls_certificate_verified +tls_peerdn +tls_sni
\end{lstlisting}
to get even more TLS information logged.

\paragraph{Client mode (outgoing):}
Exim uses opportunistic encryption in the SMTP transport by default.

Client mode settings have to be done in the configuration section of the smtp transport (driver = smtp).

If you want to use a client certificate (most server certificates can be used as client certificate, too) set
\configfile{configure.client}{152-153}{Certificate selection in Exim (Client)}
This is recommended for MTA-MTA traffic.

%If you want to limit used ciphers set
%\begin{lstlisting}
%  tls_require_ciphers = <...recommended ciphersuite...>
%\end{lstlisting}
% Exim Maintainers do not recommend ciphers. We shouldn't do so, too.
Do not limit ciphers without a very good reason. In the worst case you end up without encryption at all instead of some weak encryption. Please consult the Exim documentation if you really need to define ciphers.

\paragraph{OpenSSL:}
Exim already disables SSLv2 by default. We recommend to add
\begin{lstlisting}
openssl_options = +all +no_sslv2 +no_compression +cipher_server_preference
\end{lstlisting}
to the main configuration.

Note: +all is misleading here since OpenSSL only activates the most common workarounds. But that's how SSL\_OP\_ALL is defined.

You do not need to set dh\_parameters. Exim with OpenSSL by default uses parameter initialization with the "2048-bit MODP Group with 224-bit Prime Order Subgroup" defined in section 2.2 of RFC 5114~\cite{rfc5114} (ike23).
If you want to set your own DH parameters please read the TLS documentation of exim.


\paragraph{GnuTLS:}
GnuTLS is different in only some respects to OpenSSL:
\begin{itemize*}
  \item tls\_require\_ciphers needs a GnuTLS priority string instead of a cipher list. It is recommended to use the defaults by not defining this option. It highly depends on the version of GnuTLS used. Therefore it is not advisable to change the defaults.
  \item There is no option like openssl\_options
\end{itemize*}

\paragraph{Exim string expansion:}
Note that most of the options accept expansion strings. This way you can e.g. set cipher lists or STARTTLS advertisement conditionally. Please follow the link to the official Exim documentation to get more information.

\paragraph{Limitations:}
Exim currently (4.82) does not support elliptic curves with OpenSSL. This means that ECDHE is not used even if defined in your cipher list.
There already is a working patch to provide support:
\url{http://bugs.exim.org/show_bug.cgi?id=1397}

\subsubsection{How to test}
\begin{lstlisting}
openssl s_client -starttls smtp -crlf -connect SERVER.TLD:25
\end{lstlisting}


%% ----------------------------------------------------------------------
%\subsection{Exchange}

%\todo{FIXME: write this section}
