
PGP uses assymetric encryption to protecte a sesion key which is used to encipher a message . Additionally, it signs messages via assymetric encryption and hash functions.
Since SHA-1 as hash function has been broken already in 2005\footnote{\url{https://www.schneier.com/blog/archives/2005/02/sha1\_broken.html}}, there are a couple of settings a PGP user can change to use different hashes.

When using PGP, there are a couple of things to take care of:
\begin{itemize}
\item keylengths (see the section \ref{section:keylengths})
\item randomness (see the section \ref{section:RNGs})
\item the choice of RSA vs. DSA 
\item preferences for symmetric ciphers
\item preferences for hashing
\end{itemize}

Properly dealing with key material, passphrases and the web-of-trust  is outside of the scope of this document. The GnuPG website\footnote{\url{http://www.gnupg.org/}} has a good tutorial on PGP.

\subsubsection{keylenths}
We do not recommend any key length $\le$ 2048 bits. In fact, 4096 bits are probabaly a good choice at the time of this writing.

\subsubsection{RSA vs. DSA}
\todo{sure?}
RSA

\subsubsection{symmetric ciphers}


\subsubsection{hashing}
Tell GnuPG to not use SHA-1.

Edit \$HOME/.gnupg/gpg.conf:

\begin{lstlisting}[breaklines]
# according to:  https://www.debian-administration.org/users/dkg/weblog/48
personal-digest-preferences SHA256
cert-digest-algo SHA256
default-preference-list SHA512 SHA384 SHA256 SHA224 AES256 AES192 AES CAST5 ZLIB BZIP2 ZIP Uncompressed
\end{lstlisting}

%\subsubsection{PGP / GPG Operations}

%% Ciphering - Unciphering operations
%%% TOO COMPLEX. Make a pointer to a good GPG tutorial

%% Signing / checking signatures
%%% TOO COMPLEX. Make a pointer to a good GPG tutorial

%\subsubsection{Trusted Keys}

%%Explain that a key by himself is not trustable.  Chain of trust principle.

%%% TOO COMPLEX. Make a pointer to a good GPG tutorial

%\subsection{Available implementations and mails plugins}

%% Microsoft Windows (Symantec for Outlook? GnuPG + ....)
%%% TOO COMPLEX. Make a pointer to a good GPG tutorial

%% Linux (GnuPG + Enigmail for Thunderbird)

%%% TOO COMPLEX. Make a pointer to a good GPG tutorial
%% Mac OS X (GnuPG + GPGMail)
%%% TOO COMPLEX. Make a pointer to a good GPG tutorial


