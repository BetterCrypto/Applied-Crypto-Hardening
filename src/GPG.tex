\section{GPG/PGP - The Privacy XXX}

Welcome in the family for asymetric protocols!  GPG (the GNU version) or PGP (the commercial version) are actually the same protocol that allows you to perform 2 operations: sign and/or encrypt.

Signing is the operation to digitally proove that the text received
\begin{itemize}
    \item Has been signed by the one that own's the private key;
    \item Was NOT altered since it was signed.
\end{itemize}

Ciphering is the operation to apply an mathematical operation that will prevent anybody except the one owning the private key corresponding to the public key used to cipher the text.

With algorithm with GPG/PGP we are not using one key but actually two!  The public key and the private key.  Those key are glued forever.  One doesn't work with the other one.  Shoud you need to remember only one stuff, keep your private key top secret.  Never share it, never publish it, use a strong password to protect it.  On the contratry, share your public key as much as you can (website, business cards...).

The public key will be used by the one that receive your messages to
\begin{itemize}
    \item Check the signature of messages you send him;
    \item Cipher message that should be received by you and only you.
\end{itemize}

You will use your private key to
\begin{itemize}
    \item Sign messages;
    \item Uncopher messages you receive.
\end{itemize}


\subsection{Introduction}

\subsubsection{Ciphering}

\subsubsection{Signing}

\subsection{Key Generation}

\subsection{Operations}

\subsection{Trusted Keys}


