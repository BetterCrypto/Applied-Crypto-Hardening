\section{A note on Elliptic Curve Cryptography}
\label{section:EllipticCurveCryptography}

Elliptic Curve Cryptogaphy (simply called ECC from now on) is a branch of 
cryptography that emerged in the mid-1980ties. Like RSA and Diffie-Hellman 
its security is based on the discrete logarithm problem
\footnote{\url{http://www.mccurley.org/papers/dlog.pdf}} 
\footnote{\url{http://en.wikipedia.org/wiki/Discrete\_logarithm}}
\footnote{\url{http://mathworld.wolfram.com/EllipticCurve.html}}.
Finding the discrete logarithm of an elliptic curve from its public base
point is thought to be infeasible. This is known as the Elliptic Curve Discrete 
Logarithm Problem (ECDLP). ECC and the underlying mathematical foundation are not easy 
to understand - luckily there have been some great introductions on the topic lately
\footnote{\url{http://arstechnica.com/security/2013/10/a-relatively-easy-to-understand-primer-on-elliptic-curve-cryptography}}
\footnote{\url{https://www.imperialviolet.org/2010/12/04/ecc.html}}
\footnote{\url{http://www.isg.rhul.ac.uk/~sdg/ecc.html}}.

ECC provides for much stronger security with less computonally expensive
operations in comparison to traditional PKI algorithms. (See the section 
on keylengths to get an idea)


The security of ECC relies on the elliptic curves and curve points chosen
as parameters for the algorithm in question. Well before the NSA-leak scandal
there has been a lot of discussion regarding these parameters and their 
potential subversion. A part of the discussion involved recommended sets 
of curves and curve points chosen by different standardization bodies such 
as the National Institute of Standards and Technology (NIST) 
\footnote{\url{http://www.nist.gov}}. Which were later widely implemented 
in most common crypto libraries. Those parameters came under question 
repeatedly from cryptographers
\footnote{\url{http://cr.yp.to/talks/2013.09.16/slides-djb-20130916-a4.pdf}}
\footnote{\url{https://www.schneier.com/blog/archives/2013/09/the\_nsa\_is\_brea.html\#c1675929}}
\footnote{\url{http://crypto.stackexchange.com/questions/10263/should-we-trust-the-nist-recommended-ecc-parameters}}.
At the time of writing there is ongoing research as to the security of 
various ECC parameters
\footnote{\url{http://safecurves.cr.yp.to}}. 
Most software configured to rely on ECC (be it client or server) is
not able to promote or black-list certain curves. It is the hope of
the authors that such functionality will be deployed widely soon.
The authors of this paper include configurations and recommendations
with and without ECC - the reader may choose to adopt those settings
as he finds best suited to his environment. The authors will not make
this decision for the reader.


\textbf{A word of warning:} One should get familiar with ECC, different curves and
parameters if one chooses to adopt ECC configurations. Since there is much 
discussion on the security of ECC, flawed settings might very well compromise the 
security of the entire system!

%% mention different attacks on ECC besides flawed parameters!

