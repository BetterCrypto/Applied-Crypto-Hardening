%%% LaTeX Template: Two column article
%%%
%%% Source: http://www.howtotex.com/
%%% Feel free to distribute this template, but please keep to referal to http://www.howtotex.com/ here.
%%% Date: February 2011

%%% Preamble
\documentclass[	DIV=calc,%
				paper=a4,%
				fontsize=9pt,%
				onecolumn]{scrartcl}	 					% KOMA-article class

\usepackage{lipsum}													% Package to create dummy text



\usepackage[english]{babel}										% English language/hyphenation
\usepackage[protrusion=true,expansion=true]{microtype}				% Better typography
\usepackage{amsmath,amsfonts,amsthm}					% Math packages
\usepackage[pdftex]{graphicx}									% Enable pdflatex
%\usepackage[svgnames]{xcolor}									% Enabling colors by their 'svgnames'
\usepackage[hang, small,labelfont=bf,up,textfont=it,up]{caption}	% Custom captions under/above floats
\usepackage{epstopdf}												% Converts .eps to .pdf
\usepackage{subfig}													% Subfigures
\usepackage{booktabs}												% Nicer tables
\usepackage{fix-cm}													% Custom fontsizes

\usepackage[usenames,dvipsnames]{color}
\usepackage{float}
\usepackage{subfig}
%\usepackage{tikz}
\usepackage{acronym}
\usepackage{amsthm}
\usepackage{fancyvrb}
\usepackage{listings}


% custom changes:
\usepackage[usenames,dvipsnames,svgnames,table]{xcolor}
\usepackage{placeins}
\usepackage{hyperref}
\usepackage{draftwatermark}

% Add text symbols
\usepackage{pifont}
\newcommand{\yes}{\textcolor{green}{\ding{51}}}
\newcommand{\no}{\textcolor{red}{\ding{55}}}

% human tables
\usepackage{booktabs}
\renewcommand{\arraystretch}{1.25}

\definecolor{green}{RGB}{32,113,10}
\definecolor{orange}{RGB}{251,111,16}
\definecolor{red}{RGB}{247,56,0}
\definecolor{blue}{RGB}{0,28,128}

\bibliographystyle{plain}


\definecolor{Brown}{cmyk}{0,0.81,1,0.60}
\definecolor{OliveGreen}{cmyk}{0.64,0,0.95,0.40}
\definecolor{CadetBlue}{cmyk}{0.62,0.57,0.23,0}
\definecolor{lightlightgray}{gray}{0.9}

\lstset{
language=Bash,                             % Code langugage
basicstyle=\ttfamily,                   % Code font, Examples: \footnotesize, \ttfamily
keywordstyle=\color{OliveGreen},        % Keywords font ('*' = uppercase)
commentstyle=\color{gray},              % Comments font
%numbers=left,                           % Line nums position
%numberstyle=\tiny,                      % Line-numbers fonts
%stepnumber=1,                           % Step between two line-numbers
%numbersep=5pt,                          % How far are line-numbers from code
backgroundcolor=\color{lightlightgray}, % Choose background color
frame=none,                             % A frame around the code
tabsize=2,                              % Default tab size
captionpos=b,                           % Caption-position = bottom
breaklines=true,                        % Automatic line breaking?
breakatwhitespace=false,                % Automatic breaks only at whitespace?
showspaces=false,                       % Dont make spaces visible
showtabs=false,                         % Dont make tabls visible
columns=fixed,                          % Column format
morekeywords={__global__, __device__},  % 
}


%% \todo{} command.
%
% Outputs red TODOs in the document. Requires \usepackage{color}.
%
% Usage: \todo{Document the TODO command.}
%
% Comment out second line to disable.
\newcommand{\todo}[1]{}
\renewcommand{\todo}[1]{{\color{red} TODO: {#1}}}


%%% Custom sectioning (sectsty package)
\usepackage{sectsty}													% Custom sectioning (see below)
\allsectionsfont{%															% Change font of al section commands
	\usefont{OT1}{phv}{b}{n}%										% bch-b-n: CharterBT-Bold font
	}

\sectionfont{%																% Change font of \section command
	\usefont{OT1}{phv}{b}{n}%										% bch-b-n: CharterBT-Bold font
	}

% use more of the page
\usepackage{fullpage}

%%% Headers and footers
\usepackage{fancyhdr}												% Needed to define custom headers/footers
	\pagestyle{fancy}														% Enabling the custom headers/footers
\usepackage{lastpage}	

% Header (empty)
\lhead{}
\chead{}
\rhead{}
% Footer (you may change this to your own needs)
\lfoot{\footnotesize Applied Crypto Hardening \textbullet ~Draft}
\cfoot{}
\rfoot{\footnotesize page \thepage\ of \pageref{LastPage}}	% "Page 1 of 2"
\renewcommand{\headrulewidth}{0.0pt}
\renewcommand{\footrulewidth}{0.4pt}



%%% Creating an initial of the very first character of the content
\usepackage{lettrine}
\newcommand{\initial}[1]{%
     \lettrine[lines=3,lhang=0.3,nindent=0em]{
     				\color{DarkGoldenrod}
     				{\textsf{#1}}}{}}



%%% Title, author and date metadata
\usepackage{titling}															% For custom titles

\newcommand{\HorRule}{\color{DarkGoldenrod}%			% Creating a horizontal rule
									  	\rule{\linewidth}{1pt}%
									}

\pretitle{\vspace{-30pt} \begin{flushleft} \HorRule 
				\fontsize{36}{36} \usefont{OT1}{phv}{b}{n} \color{DarkRed} \selectfont 
				}
			\title{Applied Crypto Hardening}% \\ \vskip 0.5em \large www.bettercrypto.org}
\posttitle{\par\end{flushleft}\vskip 0.5em}

\preauthor{\begin{flushleft}
					\large \lineskip 0.5em \usefont{OT1}{phv}{b}{sl} \color{DarkRed}}

					\author{Wolfgang Breyha, David Durvaux, Tobias Dussa, L. Aaron
					Kaplan, Christian Mock, Manuel Koschuch, Adi
				Kriegisch, Ramin Sabet, Aaron Zauner, Pepi Zawodsky}
%\institute{
%FH Campus Wien
%\and
%VRVis
%\and
%CERT.at
%\and
%Karlsruhe Institute of Technology
%}


\setlength{\parindent}{0cm}

\postauthor{\footnotesize \usefont{OT1}{phv}{m}{sl} \color{Black} 
\\ \vskip 0.5em  (University of Vienna, CERT.be, KIT-CERT, CERT.at, coretec.at, FH Campus Wien, VRVis, A-Trust, azet.org, maclemon.at)
					\par\end{flushleft}\HorRule}

\date{\today}



%%% Begin document
\begin{document}
\maketitle
\thispagestyle{fancy} 			% Enabling the custom headers/footers for the first page 
% The first character should be within \initial{}
%\initial{H}\textbf{ere is some sample text to show the initial in the introductory paragraph of this template article. The color and lineheight of the initial can be modified in the preamble of this document.}

\begin{abstract}
%\section*{Abstract}


\epigraph{``Unfortunately, the computer security and cryptology communities have drivted apart over the last 25 years. Security people don't always understand the available crypto tools, and crypto people don't always understand the real-world problems.``}{-- Ross Anderson in \cite{anderson2008security}}

\vskip 2em

This guide arose out of the need for system administrators to have an
updated, solid, well researched and thought-through guide for configuring SSL,
PGP, SSH and other cryptographic tools in the post-Snowden age. Triggered by the NSA
leaks in the summer of 2013, many system administrators and IT security
officers saw the need to strengthen their encryption settings.
This guide is specifically written for these system administrators.

\vskip 0.5em

As Schneier noted\cite{Sch13},
it seems that intelligence agencies and adversaries on the Internet are not
breaking so much the mathematics of encryption per se, but rather use software
and hardware weaknesses, subvert standardization processes, plant backdoors,
rig random number generators and most of all exploit careless settings in
server configurations and encryption systems to listen in on private
communications. 

\vskip 0.5em

This guide can only address one aspect of securing our information systems:
getting the crypto settings right to the best of the authors' current
knowledge. Other attacks, as the above mentioned, require different protection
schemes which are not covered in this guide. This guide is not an introduction
to cryptography on how to use PGP nor SSL. For background information on
cryptography, cryptoanalysis, PGP and SSL we would like to refer the reader to
the the references (see chapter \ref{section:Links} and
\ref{section:Suggested_Reading}) at the end of this document.

\vskip 0.5em

The focus of this guide is merely to give current best practices for
configuring complex cipher suites and related parameters in a \textbf{copy \&
paste-able manner}. The guide tries to stay as concise as is possible for such
a complex topic as cryptography. There are many excellent guides
(\cite{ii2011ecrypt},\cite{TR02102}) and best practice documents available when
it comes to cryptography. However none of them focuses specifically on what an
average system administrator needs for hardening his or her systems' crypto
settings.

This guide tries to fill this gap.


\end{abstract}

\newpage
\tableofcontents
\newpage
\section{Disclaimer and scope}
\label{section:disclaimer}
\label{sec:disclaimer-scope}

\epigraph{``A chain is no stronger than its weakest link, and life is after all a chain''}{William James}
\epigraph{``Encryption works. Properly implemented strong crypto systems are
one of the few things that you can rely on. Unfortunately, endpoint security is
so terrifically weak that NSA can frequently find ways around it.''}{Edward
Snowden, answering questions live on the Guardian's
website~\cite{snowdenGuardianGreenwald}}


This guide specifically does not address physical security, protecting software
and hardware against exploits, basic IT security housekeeping, information
assurance techniques, traffic analysis attacks, issues with key-roll over and
key management, securing client PCs and mobile devices (theft, loss), proper
OPSec\footnote{\url{http://en.wikipedia.org/wiki/Operations_security}}, social
engineering attacks, anti-tempest~\cite{Wikipedia:Tempest} attack techniques,
protecting against different side-channel attacks (timing--, cache timing--,
differential fault analysis, differential power analysis or power monitoring
attacks), downgrade attacks, jamming the encrypted channel or other similar
attacks which are typically employed to circumvent strong encryption.  The
authors can not overstate the importance of these other techniques.  Interested
readers are advised to read about these attacks in detail since they give a lot
of insight into other parts of cryptography engineering which need to be dealt
with.\footnote{An easy to read yet very insightful recent example is the
"FLUSH+RELOAD" technique~\cite{yarom2013flush+} for leaking cryptographic keys
from one virtual machine to another via L3 cache timing attacks.}

This guide does not talk much about the well-known insecurities of trusting a
public-key infrastructure (PKI)\footnote{Interested readers are referred to
\url{https://bugzilla.mozilla.org/show_bug.cgi?id=647959} or
\url{http://www.heise.de/security/meldung/Der-ehrliche-Achmed-bittet-um-Vertrauen-1231083.html}
(german) which brings the problem of trusting PKIs right to the point}. Nor
does this text fully explain how to run your own Certificate Authority (CA). 


Most of this zoo of information security issues are addressed in the very
comprehensive book ``Security Engineering'' by Ross Anderson~\cite{anderson2008security}. 

For some experts in cryptography this text might seem too informal. However, we
strive to keep the language as non-technical as possible and fitting for our
target audience: system administrators who can collectively improve the
security level for all of their users. 



\epigraph{``Security is a process, not a product.''}{Bruce Schneier}

This guide can only describe what the authors currently
\emph{believe} to be the best settings based on their personal experience and
after intensive cross checking with literature and experts. For a complete list
of people who reviewed this paper, see the \nameref{section:Reviewers}.
Even though multiple specialists reviewed the guide, the authors can give
\emph{no guarantee whatsoever} that they made the right recommendations. Keep in
mind that tomorrow there might be new attacks on some ciphers and many of the
recommendations in this guide might turn out to be wrong. Security is a
process.


We therefore recommend that system administrators keep up to date with recent
topics in IT security and cryptography. 


In this sense, this guide is very focused on getting the cipher strings done
right even though there is much more to do in order to make a system more
secure.  We the authors, need this document as much as the reader needs it.

\subsection*{Scope}
\label{section:Scope}

In this guide, we restricted ourselves to:
\begin{itemize}
\item Internet-facing services
\item Commonly used services
\item Devices which are used in business environments (this specifically excludes XBoxes, Playstations and similar consumer devices)
\item OpenSSL 
\end{itemize}

We explicitly excluded:
\begin{itemize}
\item Specialized systems (such as medical devices, most embedded systems, etc.)
\item Wireless Access Points
\item Smart-cards/chip cards
%\item Advice on running a PKI or a CA
%\item Services which should be run only in an internal network and never face the Internet.
\end{itemize}


%%% Local Variables: 
%%% mode: latex
%%% TeX-master: "applied-crypto-hardening"
%%% End: 

%\section{Motivation}
%\label{section:Motivation}

\section{Methods}

For many years, NIST\footnote{\url{http://www.nist.gov/}} was the most prominent
standardisation institute industry would consult for recommendations in the
field of cryptography. However, the NSA leaks of 2013 showed that even certain
NIST recommendations were
subverted\footnote{\url{http://www.scientificamerican.com/article.cfm?id=nsa-nist-encryption-scandal}}
by the NSA.  Hence, we can not blindly trust NIST's recommendations on cipher
and cipher suite settings. 

We chose to collect the most well known facts about crypto-settings and let as
many trusted specialists as possible review these settings.  The review process
is completely open and done on a public mailing list. The document is available
(read-only) to the public Internet on a git server. However, write permissions
to the document are only granted to trusted people. The list of editors is made public.
Every write operation to the document is logged via the ``git'' version
control system and thus can be traced back to a specific author.  We do not
trust an unknown git server. 

Public peer-review / ``multiple eyes'' checking our recommendation is the best
strategy we can imagine.




\section{Scope}

We are only analyzing...
* internet serving devices
* ...



\section{Public Key Infrastructures}
\label{section:PKIs}

Public-Key Infrastructures aim to provide a way to simplify the verification of
a certificates trustworthiness.  For this, certificate authorities (CAs) are
used to create a signature chain from the root CA down to the server (or client).
Accepting a CA as a generally-trusted mediator solves the trust-scaling problem
at the cost of introducing an actor that magically is more trustworthy.

This section deals with settings related to trusting CAs. However, our main
recommendations for PKIs is: if you are able to run your own PKI and disable
any other CA, do so. This makes sense most in environments where any machine-to-machine
communication system compatibility with external entities is not an issue.
%% azet:
%% this needs discussion! self-signed certificates simply do not work in practices
%% for real-world scenarios - i.e. websites that actually serve a lot of people

A good background on PKIs can be found in 
\footnote{\url{https://developer.mozilla.org/en/docs/Introduction_to_Public-Key_Cryptography}}
\footnote{\url{http://cacr.uwaterloo.ca/hac/about/chap8.pdf}}
\footnote{\url{http://www.verisign.com.au/repository/tutorial/cryptography/intro1.shtml}}
.

\todo{ts: Background and Configuration (EMET) of Certificate Pinning, TLSA integration, 
  When to use self-signed certificates, how to get certificates from public CA authorities 
  (CACert, StartSSL), Best-practices how to create a CA and how to generate private keys/CSRs, 
  Discussion about OCSP and CRLs. TD: Useful Firefox plugins: CipherFox, Conspiracy, Perspectives.}


%``Certification
%Policy''\footnote{\url{http://en.wikipedia.org/wiki/Certificate_Policy}}
%(CA)

\section{A note on Elliptic Curve Cryptography}
\label{section:EllipticCurveCryptography}

Elliptic Curve Cryptogaphy (simply called ECC from now on) is a branch of 
cryptography that emerged in the mid-1980ties. Like RSA and Diffie-Hellman 
its security is based on the discrete logarithm problem
\footnote{\url{http://www.mccurley.org/papers/dlog.pdf}} 
\footnote{\url{http://en.wikipedia.org/wiki/Discrete\_logarithm}}
\footnote{\url{http://mathworld.wolfram.com/EllipticCurve.html}}.
Finding the discrete logarithm of an elliptic curve from its public base
point is thought to be infeasible. This is known as the Elliptic Curve Discrete 
Logarithm Problem (ECDLP). ECC and the underlying mathematical foundation are not easy 
to understand - luckily there have been some great introductions on the topic lately
\footnote{\url{http://arstechnica.com/security/2013/10/a-relatively-easy-to-understand-primer-on-elliptic-curve-cryptography}}
\footnote{\url{https://www.imperialviolet.org/2010/12/04/ecc.html}}
\footnote{\url{http://www.isg.rhul.ac.uk/~sdg/ecc.html}}.

ECC provides for much stronger security with less computonally expensive
operations in comparison to traditional PKI algorithms. (See the section 
on keylengths to get an idea)


The security of ECC relies on the elliptic curves and curve points chosen
as parameters for the algorithm in question. Well before the NSA-leak scandal
there has been a lot of discussion regarding these parameters and their 
potential subversion. A part of the discussion involved recommended sets 
of curves and curve points chosen by different standardization bodies such 
as the National Institute of Standards and Technology (NIST) 
\footnote{\url{http://www.nist.gov}}. Which were later widely implemented 
in most common crypto libraries. Those parameters came under question 
repeatedly from cryptographers
\footnote{\url{http://cr.yp.to/talks/2013.09.16/slides-djb-20130916-a4.pdf}}
\footnote{\url{https://www.schneier.com/blog/archives/2013/09/the\_nsa\_is\_brea.html\#c1675929}}
\footnote{\url{http://crypto.stackexchange.com/questions/10263/should-we-trust-the-nist-recommended-ecc-parameters}}.
At the time of writing there is ongoing research as to the security of 
various ECC parameters
\footnote{\url{http://safecurves.cr.yp.to}}. 
Most software configured to rely on ECC (be it client or server) is
not able to promote or black-list certain curves. It is the hope of
the authors that such functionality will be deployed widely soon.
The authors of this paper include configurations and recommendations
with and without ECC - the reader may choose to adopt those settings
as he finds best suited to his environment. The authors will not make
this decision for the reader.


\textbf{A word of warning:} One should get familiar with ECC, different curves and
parameters if one chooses to adopt ECC configurations. Since there is much 
discussion on the security of ECC, flawed settings might very well compromise the 
security of the entire system!

%% mention different attacks on ECC besides flawed parameters!


\section{Keylengths}
\label{section:keylengths}

Recommendations on keylengths need to be adapted regularly. Since this document
is static, we will rather refer to existing publications and websites.
Recommending the right key length is a hit-and-miss issue.

\url{http://www.keylength.com/} offers a good overview of approximations for
key size security based on recommendations by standardization bodies and
academic publications.

\begin{itemize}
\item For RSA cryptography we consider any key length below 2048 bits to be
deprectated at the time of this writing.  
\item For ECC we consider key length below
224 bits to be deprectated.  
\item For symmetric algorithms, consider anything below
128 bits to be deprecated.
\end{itemize}


One special remark is necessary for 3DES: here we want to note that it
theoretically has 168 bit security, however based on the NIST Special
Publication 800-57
\footnote{\url{http://csrc.nist.gov/publications/PubsSPs.html\#800-57-part1},
pages 63 and 64}, it is clear that 3DES is only considered 80 bits / 112 bits.






\section{Random Number Generators}
\label{section:RNGs}

% This section was authored by Ralf Schlatterbeck (Ralf Schlatterbeck <rsc@runtux.com>)

\epigraph{``The generation of random numbers is too important to be left to chance.''}{Robert R. Coveyou}


\begin{figure}[h]
  \centering
  \includegraphics[width=0.4\textwidth]{img/random_number.png}
  \caption{xkcd, source: \url{https://imgs.xkcd.com/comics/random_number.png}, license: CC-BY-NC}
  \label{fig:dilbertRNG}
\end{figure}



A good source of random numbers is essential for many crypto
operations. The key feature of a good random number generator is the
non-predictability of the generated numbers. This means that hardware
support for generating entropy is essential.


Hardware random number generators in operating systems or standalone
components collect entropy from various random events mostly by using
the (low bits of the) time an event occurs as an entropy source. The
entropy is merged into an entropy pool and in some implementations there
is some bookkeeping about the number of random bits available.

\subsection{When random number generators fail}

Random number generators can fail -- returning predictable non-random
numbers -- if not enough entropy is available when random numbers should
be generated.

This typically occurs for embedded devices and virtual machines.
Embedded devices lack some entropy sources other devices have, e.g.:

\begin{itemize*}
  \item No persistent clock, so boot-time is not contributing to the
    initial RNG state
  \item No hard-disk: No entropy from hard-disk timing, no way to store
    entropy between reboots
\end{itemize*}

Virtual machines emulate some hardware components so that the
generated entropy is over-estimated. The most critical component that
has been shown to return wrong results in an emulated environment is the
timing source~\cite{Eng11,POL11}.

Typically the most vulnerable time where low-entropy situations occur is
shortly after a reboot. Unfortunately many operating system installers
create cryptographic keys shortly after a reboot~\cite{HDWH12}.

Another problem is that OpenSSL seeds its internal random generator only
seldomly from the hardware random number generator of the operating
system. This can lead to situations where a daemon that is started at a
time when entropy is low keeps this low-entropy situation for hours
leading to predictable session keys~\cite{HDWH12}.

\subsection{Linux}
\label{subsec:RNG-linux}

\todo{Other architectures, BSD, Windows?}

On Linux there are two devices that return random bytes when read; the
\verb+/dev/random+ can block until sufficient entropy has been collected
while \verb+/dev/urandom+ will not block and return whatever (possibly
insufficient) entropy has been collected so far.

Unfortunately most crypto implementations are using \verb+/dev/urandom+
and can produce predictable random numbers if not enough entropy has
been collected~\cite{HDWH12}.

Linux supports the injection of additional entropy into the entropy pool
via the device \verb+/dev/random+. On the one hand this is used for
keeping entropy across reboots by storing output of /dev/random into a
file before shutdown and re-injecting the contents during the boot
process. On the other hand this can be used for running a secondary
entropy collector to inject entropy into the kernel entropy pool.

On Linux you can check how much entropy is available with the command:
\begin{lstlisting}
$ cat /proc/sys/kernel/random/entropy_avail
\end{lstlisting}

%% specifics for libraries
%% Openssl uses /dev/urandom. See the paper: https://factorable.net/weakkeys12.conference.pdf (section 5.2)
%% What about other libs? 
%% What about other OSes? 


\subsection{Recommendations}

To avoid situations where a newly deployed server doesn't have enough
entropy it is recommended to generate keys (e.g. for SSL or SSH) on
a system with a sufficient amount of entropy available and transfer the generated keys
to the server.  This is especially advisable for small embedded devices
or virtual machines.

For embedded devices and virtual machines deploying additional userspace
software that generates entropy and feeds this to kernel entropy pool
(e.g. by writing to \verb+/dev/random+ on Linux) is recommended. Note
that only a process with root rights can update the entropy counters in the
kernel; non-root or user processes can still feed entropy to the pool but
cannot update the counters~\cite{Wikipedia:/dev/random}.

For Linux the \verb+haveged+
implementation~\cite{HAV13a} based on the HAVEGE~\cite{SS03}
strong random number generator currently looks like the best choice. It
can feed its generated entropy into the kernel entropy pool and recently
has grown a mechanism to monitor the quality of generated random
numbers~\cite{HAV13b}. The memory footprint may be too high for small
embedded devices, though.

For systems where -- during the lifetime of the keys -- it is expected
that low-entropy situations occur, RSA keys should be preferred over DSA
keys: For DSA, if there is ever insufficient entropy at the time keys
are used for signing this may lead to repeated ephemeral keys. An
attacker who can guess an ephemeral private key used in such a signature
can compromise the DSA secret key.
For RSA this can lead to discovery of encrypted plaintext or forged
signatures but not to the compromise of the secret key~\cite{HDWH12}.


\section{Cipher suites}
\label{section:CipherSuites}
\todo{team: section \ref{section:CipherSuites} is currently a bit messy. Re-do it}


\subsection{Architectural overview }
\label{subsection:architecture}
%%\subsection{Architectural overview }

This section defines some terms which will be used throughout this guide.


A cipher suite is a standardised collection of key exchange algorithms, encryption 
algorithms (ciphers) and Message authentication codes (MAC) that provides authenticated 
encryption schemes. It consists of the following components:

\begin{description}

\item{\it Key exchange protocol:}
``An (interactive) key exchange protocol is a method whereby parties who do not 
share any secret information can generate a shared, secret key by communicating 
over a public channel. The main property guaranteed here is that an 
eavesdropping adversary who sees all the messages sent over the communication 
line does not learn anything about the resulting secret key.'' \cite{katz2008introduction}

Example: \texttt{DHE}

\item{\it Authentication:}
The client authenticates the server by its certificate. Optionally the server 
may authenticate the client certificate.

Example: \texttt{RSA}

\item{\it Cipher:}
The cipher is used to encrypt the message stream. It also contains the key size
and mode used by the suite.

Example: \texttt{AES256}

\item{\it Message authentication code (MAC):}
A MAC ensures that the message has not been tampered with (integrity).

Examples: \texttt{SHA256}

\item{\it Authenticated Encryption with Associated Data (AEAD):}
AEAD is a optional scheme and class for authenticated encryption block-cipher modes
which take care of encryption as well as authentication (e.g. GCM, CCM mode). 

Examples: \texttt{AES256-GCM}



\begin{figure}[h]
\makebox[\textwidth]{
\framebox[1.1\width]{ \texttt{DHE} }--\framebox[1.1\width]{ \texttt{RSA} }--\framebox[1.1\width]{ \texttt{AES256} }--\framebox[1.1\width]{ \texttt{SHA256} } }
\caption{Composition of a typical cipher string}
\end{figure}

\item {\textbf{NOTE:}} there are two naming schemes for cipher strings -- IANA names (see section \ref{section:Links}) and the more well known OpenSSL names. In this document we will always use OpenSSL names unless a specific service uses IANA names.

\end{description}



\subsection{Forward Secrecy}
\label{subsection:PFS}
%%\subsection{Forward Secrecy}
Forward Secrecy or Perfect Forward Secrecy is a property of a cipher suite 
that ensures confidentiality even if the server key has been compromised.
Thus if traffic has been recorded it can not be decrypted even if an adversary
has got hold of the server key
\footnote{\url{http://en.wikipedia.org/wiki/Forward\_secrecy}}
\footnote{\url{https://www.eff.org/deeplinks/2013/08/pushing-perfect-forward-secrecy-important-web-privacy-protection}}. 


\subsection{Recommended cipher suites}
\label{section:recommendedciphers}
%%\subsection{Recommended cipher suites}

In principle system administrators who want to improve their communication security
have to make a difficult decision between effectively locking out some users and
keeping high cipher suite security while supporting as many users as possible.
The website \url{https://www.ssllabs.com/} gives administrators and security engineers
a tool to test their setup and compare compatibility with clients. The authors made
use of ssllabs.com to arrive at a set of cipher suites which we will recommend
throughout this document.

%\textbf{Caution: these settings can only represent a subjective
%choice of the authors at the time of writing. It might be a wise choice to
%select your own and review cipher suites based on the instructions in section
%\ref{section:ChoosingYourOwnCipherSuites}}.


\subsubsection{Configuration A: Strong ciphers, fewer clients}

At the time of writing, our recommendation is to use the following set of strong cipher
suites which may be useful in an environment where one does not depend on many,
different clients and where compatibility is not a big issue.  An example
of such an environment might be machine-to-machine communication or corporate
deployments where software that is to be used can be defined without restrictions.


We arrived at this set of cipher suites by selecting:

\begin{itemize*}
  \item TLS 1.2
  \item Perfect forward secrecy / ephemeral Diffie Hellman
  \item strong MACs (SHA-2) or
  \item GCM as Authenticated Encryption scheme
\end{itemize*}

This results in the OpenSSL string:
\ttbox{EDH+aRSA+AES256:EECDH+aRSA+AES256:!SSLv3'}

%$\implies$ resolves to 
%
%\begin{verbatim}
%openssl ciphers -V $string
%\end{verbatim}



%\todo{make a column for cipher chaining mode} --> not really important, is it?
\ctable[caption={Configuration A ciphers},label=tab:conf-a]{lllllll}{}{%
\FL \textbf{ID}   & \textbf{OpenSSL Name}       & \textbf{Version} & \textbf{KeyEx} & \textbf{Auth} & \textbf{Cipher} & \textbf{MAC}
\ML \texttt{0x009F} & DHE-RSA-AES256-GCM-SHA384   & TLSv1.2          & DH             &  RSA          & AESGCM(256)     & AEAD
\NN \texttt{0x006B} & DHE-RSA-AES256-SHA256       & TLSv1.2          & DH             &  RSA          & AES(256) (CBC)  & SHA256
\NN \texttt{0xC030} & ECDHE-RSA-AES256-GCM-SHA384 & TLSv1.2          & ECDH           &  RSA          & AESGCM(256)     & AEAD
\NN \texttt{0xC028} & ECDHE-RSA-AES256-SHA384     & TLSv1.2          & ECDH           &  RSA          & AES(256) (CBC)  & SHA384
\LL}

\paragraph*{Compatibility:}

At the time of this writing only Win 7 and Win 8.1 crypto stack,
OpenSSL $\ge$ 1.0.1e, Safari 6 / iOS 6.0.1 and Safari 7 / OS X 10.9
are covered by that cipher string.

% XXX author: (Adi) this depends on the chosing your own cipher chapter XXX
%In case you need to support other/different clients, see information
%about choosing your own cipher string in section
%\ref{section:ChoosingYourOwnCipherSuites}.

\subsubsection{Configuration B: Weaker ciphers but better compatibility}

In this section we propose a slightly weaker set of cipher suites.  For
example, there are known weaknesses for the SHA-1 hash function that is
included in this set.  The advantage of this set of cipher suites is not only
better compatibility with a broad range of clients, but also less computational
workload on the provisioning hardware.


\textbf{All examples in this publication use Configuration B}.\\

We arrived at this set of cipher suites by selecting:

\begin{itemize*}
  \item TLS 1.2, TLS 1.1, TLS 1.0
  \item allowing SHA-1 (see the comments on SHA-1 in section \ref{section:SHA})
\end{itemize*}

This results in the OpenSSL string:
%
%'EDH+CAMELLIA:EDH+aRSA:EECDH+aRSA+AESGCM:EECDH+aRSA+SHA384:EECDH+aRSA+SHA256:EECDH:+CAMELLIA256:+AES256:+CAMELLIA128:+AES128:+SSLv3:!aNULL:!eNULL:!LOW:!3DES:!MD5:!EXP:!PSK:!SRP:!DSS:!RC4:!SEED:!ECDSA:CAMELLIA256-SHA:AES256-SHA:CAMELLIA128-SHA:AES128-SHA'
\ttbox{\cipherStringB}

\todo{make a column for cipher chaining mode}
\ctable[pos=ht,caption={Configuration B ciphers},label=tab:conf-b]{lllllll}{}{%
\FL \textbf{ID}   & \textbf{OpenSSL Name}       & \textbf{Version} & \textbf{KeyEx} & \textbf{Auth} & \textbf{Cipher} & \textbf{MAC}
\ML \texttt{0x009F} & DHE-RSA-AES256-GCM-SHA384   & TLSv1.2          & DH             & RSA           & AESGCM(256)     & AEAD
\NN \texttt{0x006B} & DHE-RSA-AES256-SHA256       & TLSv1.2          & DH             & RSA           & AES(256)        & SHA256
\NN \texttt{0xC030} & ECDHE-RSA-AES256-GCM-SHA384 & TLSv1.2          & ECDH           & RSA           & AESGCM(256)     & AEAD
\NN \texttt{0xC028} & ECDHE-RSA-AES256-SHA384     & TLSv1.2          & ECDH           & RSA           & AES(256)        & SHA384
\NN \texttt{0x009E} & DHE-RSA-AES128-GCM-SHA256   & TLSv1.2          & DH             & RSA           & AESGCM(128)     & AEAD
\NN \texttt{0x0067} & DHE-RSA-AES128-SHA256       & TLSv1.2          & DH             & RSA           & AES(128)        & SHA256
\NN \texttt{0xC02F} & ECDHE-RSA-AES128-GCM-SHA256 & TLSv1.2          & ECDH           & RSA           & AESGCM(128)     & AEAD
\NN \texttt{0xC027} & ECDHE-RSA-AES128-SHA256     & TLSv1.2          & ECDH           & RSA           & AES(128)        & SHA256
\NN \texttt{0x0088} & DHE-RSA-CAMELLIA256-SHA     & SSLv3            & DH             & RSA           & Camellia(256)   & SHA1
\NN \texttt{0x0039} & DHE-RSA-AES256-SHA          & SSLv3            & DH             & RSA           & AES(256)        & SHA1
\NN \texttt{0xC014} & ECDHE-RSA-AES256-SHA        & SSLv3            & ECDH           & RSA           & AES(256)        & SHA1
\NN \texttt{0x0045} & DHE-RSA-CAMELLIA128-SHA     & SSLv3            & DH             & RSA           & Camellia(128)   & SHA1
\NN \texttt{0x0033} & DHE-RSA-AES128-SHA          & SSLv3            & DH             & RSA           & AES(128)        & SHA1
\NN \texttt{0xC013} & ECDHE-RSA-AES128-SHA        & SSLv3            & ECDH           & RSA           & AES(128)        & SHA1
\NN \texttt{0x0084} & CAMELLIA256-SHA             & SSLv3            & RSA            & RSA           & Camellia(256)   & SHA1
\NN \texttt{0x0035} & AES256-SHA                  & SSLv3            & RSA            & RSA           & AES(256)        & SHA1
\NN \texttt{0x0041} & CAMELLIA128-SHA             & SSLv3            & RSA            & RSA           & Camellia(128)   & SHA1
\NN \texttt{0x002F} & AES128-SHA                  & SSLv3            & RSA            & RSA           & AES(128)        & SHA1
\LL}
\paragraph*{Compatibility: }

Note that these cipher suites will not work with Windows XP's crypto stack (e.g. IE, Outlook),
%%Java 6, Java 7 and Android 2.3. Java 7 could be made compatible by installing the "Java 
%%Cryptography Extension (JCE) Unlimited Strength Jurisdiction Policy Files"
%%(JCE) \footnote{\url{http://www.oracle.com/technetwork/java/javase/downloads/jce-7-download-432124.html}}.
We could not verify yet if installing JCE also fixes the Java 7
DH-parameter length limitation (1024 bit). 
\todo{do that!}

\paragraph*{Explanation: }

For a detailed explanation of the cipher suites chosen, please see
\ref{section:ChoosingYourOwnCipherSuites}. In short, finding a single perfect cipher
string is practically impossible and there must be a tradeoff between compatibility and security.
On the one hand there are mandatory and optional ciphers defined in a few RFCs, 
on the other hand there are clients and servers only implementing subsets of the 
specification.

Straightforwardly, the authors wanted strong ciphers, forward secrecy
\footnote{\url{http://nmav.gnutls.org/2011/12/price-to-pay-for-perfect-forward.html}}
and the best client compatibility possible while still ensuring a cipher string that can be
used on legacy installations (e.g. OpenSSL 0.9.8).

Our recommended cipher strings are meant to be used via copy and paste and need to work
"out of the box".

\begin{itemize*}
  \item TLSv1.2 is preferred over TLSv1.0 (while still providing a useable cipher
      string for TLSv1.0 servers).
  \item AES256 and CAMELLIA256 count as very strong ciphers at the moment.
  \item AES128 and CAMELLIA128 count as strong ciphers at the moment
  \item DHE or ECDHE for forward secrecy
  \item RSA as this will fit most of today's setups
  \item AES256-SHA as a last resort: with this cipher at the end, even server
      systems with very old OpenSSL versions will work out of the box (version 0.9.8 for example does not
      provide support for ECC and TLSv1.1 or above). \newline
      Note however that this cipher suite will not provide forward secrecy. It
      is meant to provide the same client coverage (eg. support Microsoft crypto
      libraries) on legacy setups.
\end{itemize*}



%\subsection{Known insecure and weak cipher suites}
%%%\subsection{Known insecure and weak cipher suites}
\todo{PG: please write this section. List all known broken, obsolete, weak and insecure cipher suites . Or even better: find the best site which keeps track of outdated cipher suites and simply reference it. We do not want to maintain such a list ourselves!}

Ciphers with 112bit or less are considered weak and aren't recommended. Note that
\texttt{3DES} provides only 112bit of security
\footnote{\url{http://csrc.nist.gov/publications/PubsSPs.html\#800-57-part1}}.


\subsection{Compatibility}
\label{subsection:compatibility}
%\subsection{Compatibility}
%\label{section:compatibility}
\todo{write this section. The idea here is to first document which server (and openssl) version we assumed. Once these parameters are fixed, we then list all clients which are supported for Variant A) and B). Therefore we can document compatibilities to some extent. The sysadmin can then choose roughly what he looses or gains by omitting certain cipher suites.}



\subsection{Choosing your own cipher suites}
\label{section:ChoosingYourOwnCipherSuites}
%%\subsection{Choosing your own cipher suites}
\label{section:ChoosingYourOwnCipherSuites}

\todo{ Adi...  you want to describe how to make your own selection of cipher suites here.}

SSL/TLS cipher suites consist of a key exchange mechanism, an authentication, a
stream cipher (or a block cipher with a chaining mode) and a message authentication
mechanism.

Many of those mechanisms are interchangeable like the key exchange in this example:
\texttt{ECDHE-RSA-AES256-GCM-SHA384} and \texttt{DHE-RSA-AES256-GCM-SHA384}.
To provide a decent level of security, all algorithms need to be safe (subject to
the disclaimer in section \ref{section:disclaimer}).

Note: There are some very weak cipher suites in about every crypto library, most of
them for historic reasons like the crypto export embargo
\footnote{\url{http://en.wikipedia.org/wiki/Export_of_cryptography_in_the_United_States}}.
For the following chapter support of those is assumed to be disabled by having
\texttt{!EXP:!LOW:!NULL} as part of the cipher string.

\todo{Team: do we need references for all cipher suites considered weak?}

\subsubsection{key exchange}

Many algorithms allow a secure key exchange. Among those are RSA, DSA, DH, EDH, ECDSA,
ECDH, EECDH and a few others. During the key exchange, keys for authentication and for
encryption are exchanged. For RSA and DSA those keys are the same.

\todo{explain this section}

\begin{center}
\begin{tabular}{| l | l | l | l |}
    \toprule
 & \textbf{Key}  & \textbf{\cellcolor{orange}EC}  & \textbf{\cellcolor{green}ephemeral} \\ \cmidrule(lr){1-4}
    \cellcolor{red}    RSA   & RSA  & \cellcolor{green}no   & \cellcolor{red} no         \\
    \cellcolor{red}    DH    & RSA  & \cellcolor{green}no   & \cellcolor{red} no         \\
    \cellcolor{green}  EDH   & RSA  & \cellcolor{green}no   & \cellcolor{green} yes      \\
    \cellcolor{red}    ECDH  & both & \cellcolor{orange}yes & \cellcolor{red} no         \\
    \cellcolor{orange} EECDH & both & \cellcolor{orange}yes & \cellcolor{green} yes      \\
    \cellcolor{red}    DSA   & DSA  & \cellcolor{green}no   & \cellcolor{red} no         \\
    \cellcolor{red}    ECDSA & DSA  & \cellcolor{orange}yes & \cellcolor{red} no         \\
\bottomrule
\end{tabular}
%\\
%\\
%disabled: \texttt{!PSK:!SRP}
\end{center}

\textbf{Ephemeral Key Exchange} uses different keys for authentication (the server's RSA
key) and encryption (a randomly created key). This advantage is called ``Forward
Secrecy'' and means that even recorded traffic cannot be decrypted later when someone
gets the server key. \\
All ephemeral key exchange mechanisms base on Diffie-Hellman algorithm and require
pre-generated Diffe-Hellman parameter (which allow fast ephemeral key generation). It
is important to note that the Diffie-Hellman parameters need to be at least as strong
(speaking in number of bits) as the RSA host key. \todo{TODO: reference!}


\textbf{Elliptic Curves}\ref{section:EllipticCurveCryptography} required by current TLS
standards only consist of the so-called NIST-curves (\texttt{secp256r1} and
\texttt{secp384r1}) which may be weak because the parameters that led to their generation
weren't properly explained (by the NSA). \\
Disabling support for Elliptic Curves leads to no ephemeral key exchange being available
for the Windows platform. When you decide to use Elliptic Curves despite the uncertainty,
make sure to at least use the stronger curve of the two supported by all clients
(\texttt{secp384r1}).


Other key exchange mechanisms like Pre-Shared Key (PSK) or Secure Remote Password
(SRP) are irrelevant for regular SSL/TLS use.

\subsubsection{authentication}

RSA, DSA, DSS, ECDSA, ECDH, FORTEZZA(?).

Other authentication mechanisms like Pre Shared Keys aren't used in SSL/TLS: \texttt{!PSK:!aNULL}

\subsubsection{encryption}

AES, CAMELLIA, SEED, ARIA(?), FORTEZZA(?)...

Other ciphers like IDEA, RC2, RC4, 3DES or DES are weak and therefor not recommended:
\texttt{!DES:!3DES:!RC2:!RC4:!eNULL}

\subsubsection{message authentication}

SHA-1 (SHA), SHA-2 (SHA256, SHA384), AEAD

Note that SHA-1 is considered broken and should not be used. SHA-1 is however a the
only still available message authentication mechanism supporting TLS1.0/SSLv3. Without
SHA-1 most clients will be locked out.

Other hash functions like MD2, MD4 or MD5 are unsafe and broken: \texttt{!MD2:!MD4:!MD5}

\subsubsection{combining cipher strings}
%% reference 'man ciphers' and 'openssl ciphers' and show some simple examples
%% VERY IMPORTANT: hint at the IANA-list and the differences in implementations

\todo{ Adi...  The text below was simply the old text, still left here for reference.}

%%% NOTE: we do not need to list this all here, can move to an appendix
%At the time of this writing, SSL is defined in RFCs: 	
%
%\begin{itemize}
%\item RFC2246 - TLS1.0		
%\item RFC3268 - AES		
%\item RFC4132 - Camelia		
%\item RFC4162 - SEED		
%\item RFC4279 - PSK		
%\item RFC4346 - TLS 1.1		
%\item RFC4492 - ECC		
%\item RFC4785 - PSK\_NULL		
%\item RFC5246 - TLS 1.2		
%\item RFC5288 - AES\_GCM		
%\item RFC5289 - AES\_GCM\_SHA2\_ECC		
%\item RFC5430 - Suite B		
%\item RFC5487 - GCM\_PSK		
%\item RFC5489 - ECDHE\_PSK		
%\item RFC5932 - Camelia		
%\item RFC6101 - SSL 3.0		
%\item RFC6209 - ARIA		
%\item RFC6367 - Camelia		
%\item RFC6655 - AES\_CCM		
%\item RFC7027 - Brainpool Curves		
%\end{itemize}

\subsubsection{Overview of SSL Server settings}


Most Server software (Webservers, Mail servers, etc.) can be configured to prefer certain cipher suites over others. 
We followed the recommendations by Ivan Ristic's SSL/TLS Deployment Best Practices\footnote{\url{https://www.ssllabs.com/projects/best-practices/index.html}} document (see section 2.2 "Use Secure Protocols") and arrived at a list of recommended cipher suites for SSL enabled servers.

Following Ivan Ristic's adivce we arrived at a categorisation of cipher suites.

\begin{center}
\begin{tabular}{lllll}
\cmidrule[\heavyrulewidth]{2-5}
& \textbf{Version}   & \textbf{KeyEx} & \textbf{Cipher}    & \textbf{MAC}       \\\cmidrule(lr){2-5}
\cellcolor{green}prefer  & TLS 1.2   & DHE\_DSS   & AES\_256\_GCM   & SHA384        \\
    &   & DHE\_RSA   & AES\_256\_CCM   & SHA256        \\
    &   & ECDHE\_ECDSA   & AES\_256\_CBC   &       \\
    &   & ECDHE\_RSA &   &       \\ 
    &   &   &   &       \\
\cellcolor{orange}consider    & TLS 1.1   & DH\_DSS    & AES\_128\_GCM   & SHA       \\
    & TLS 1.0   & DH\_RSA    & AES\_128\_CCM   &       \\
    &   & ECDH\_ECDSA    & AES\_128\_CBC   &       \\ 
    &   & ECDH\_RSA  & CAMELLIA\_256\_CBC  &       \\
    &   & RSA   & CAMELLIA\_128\_CBC  &       \\
    &   &   &   &       \\
\cellcolor{red}avoid   
& SSL 3.0   & NULL  & NULL  & NULL      \\
    &   & DH\_anon   & RC4\_128   & MD5       \\
    &   & ECDH\_anon & 3DES\_EDE\_CBC  &       \\
    &   &   & DES\_CBC   &       \\
    &   &   &   &       \\
\cellcolor{blue}{\color{white}special }
&   & PSK   & CAMELLIA\_256\_GCM  &       \\
    &   & DHE\_PSK   & CAMELLIA\_128\_GCM  &       \\
    &   & RSA\_PSK   & ARIA\_256\_GCM  &       \\
    &   & ECDHE\_PSK & ARIA\_256\_CBC  &       \\
    &   &   & ARIA\_128\_GCM  &       \\
    &   &   & ARIA\_128\_CBC  &       \\
    &   &   & SEED  &       \\
\cmidrule[\heavyrulewidth]{2-5}
\end{tabular}
\end{center}

A remark on the ``consider'' section: the BSI (Federal office for information security, Germany) recommends in its technical report TR-02102-2\footnote{\url{https://www.bsi.bund.de/SharedDocs/Downloads/DE/BSI/Publikationen/TechnischeRichtlinien/TR02102/BSI-TR-02102-2_pdf.html}} to \textbf{avoid} non-ephemeral\footnote{Ephemeral keys are session keys which are destroyed upon termination of the encrypted session. In TLS/SSL, they are realized by the DHE cipher suites. } keys for any communication which might contain personal or sensitive data. In this document, we follow BSI's advice and therefore only keep cipher suites containing (EC)DH\textbf{E} (ephemeral) variants. System administrators, who can not use forward secrecy can still use the cipher suites in the ``consider'' section. We however, do not recommend them in this document.

%% NOTE: s/forward secrecy/perfect forward secrecy???

Note that the entries marked as ``special'' are cipher suites which are not common to all clients (webbrowsers etc).


\subsubsection{Tested clients}
 
Next we tested the cipher suites above on the following clients:

%% NOTE: we need to test with more systems!!
\begin{itemize}
\item Chrome 30.0.1599.101 Mac OS X 10.9
\item Safari 7.0 Mac OS X 10.9
\item Firefox 25.0 Mac OS X 10.9
\item Internet Explorer 10 Windows 7
\item Apple iOS 7.0.3
\end{itemize}


The result of testing the cipher suites with these clients gives us a preference order as shown in table \ref{table:prefOrderCipherSuites}. 
Should a client not be able to use a specific cipher suite, it will fall back to the next possible entry as given by the ordering.

\begin{table}[h]
\centering\small
    \begin{tabular}{cllcccc}
    \toprule
    \textbf{Pref}   & \textbf{Cipher Suite}                            & \textbf{ID}   & \multicolumn{4}{l}{\textbf{Supported by}}\\ 
    \cmidrule(lr){4-7}
                    & \textbf{OpenSSL Name}                            &               & Chrome & FF   & IE   & Safari \\
    \cmidrule(lr){1-7}
    \phantom{0}1    & \verb|TLS_DHE_RSA_WITH_AES_256_GCM_SHA384|     & \verb|0x009f| & \no    & \no  & \no  & \no    \\
                    & \verb|DHE-RSA-AES256-GCM-SHA384|                      &               & &&&\\\rowcolor{lightlightgray}
    \phantom{0}2    & \verb|TLS_ECDHE_ECDSA_WITH_AES_256_CBC_SHA384| & \verb|0xC024| & \no    & \no  & \no  & \yes   \\\rowcolor{lightlightgray}
                    & \verb|ECDHE-ECDSA-AES256-SHA384|                      &               & &&&\\
    \phantom{0}3    & \verb|TLS_ECDHE_RSA_WITH_AES_256_CBC_SHA384|   & \verb|0xC028| & \no    & \no  & \no  & \yes   \\
                    & \verb|ECDHE-RSA-AES256-SHA384|                        &               & &&&\\\rowcolor{lightlightgray}
    \phantom{0}4    & \verb|TLS_DHE_RSA_WITH_AES_256_CBC_SHA256|     & \verb|0x006B| & \yes   & \no  & \no  & \yes   \\\rowcolor{lightlightgray}
                    & \verb|DHE-RSA-AES256-SHA256|                          &               & &&&\\
    \phantom{0}5    & \verb|TLS_ECDHE_ECDSA_WITH_AES_256_CBC_SHA|    & \verb|0xC00A| & \yes   & \yes & \yes & \yes   \\
                    & \verb|ECDHE-ECDSA-AES256-SHA|                         &               & &&&\\\rowcolor{lightlightgray}
    \phantom{0}6    & \verb|TLS_ECDHE_RSA_WITH_AES_256_CBC_SHA|      & \verb|0xC014| & \yes   & \yes & \yes & \yes   \\\rowcolor{lightlightgray}
                    & \verb|ECDHE-RSA-AES256-SHA|                           &               & &&&\\
    \phantom{0}7    & \verb|TLS_DHE_RSA_WITH_AES_256_CBC_SHA|        & \verb|0x0039| & \yes   & \yes & \no  & \yes   \\
                    & \verb|DHE-RSA-AES256-SHA|                             &               & &&&\\\rowcolor{lightlightgray}
    \phantom{0}8    & \verb|TLS_DHE_DSS_WITH_AES_256_CBC_SHA|        & \verb|0x0038| & \no    & \yes & \yes & \no    \\\rowcolor{lightlightgray}
                    & \verb|DHE-DSS-AES256-SHA|                             &               & &&&\\
    \phantom{0}9    & \verb|TLS_DHE_RSA_WITH_CAMELLIA_256_CBC_SHA|   & \verb|0x0088| & \no    & \yes & \no  & \no    \\
                    & \verb|DHE-RSA-CAMELLIA256-SHA|                        &               & &&&\\\rowcolor{lightlightgray}
    \phantom{}10    & \verb|TLS_DHE_DSS_WITH_CAMELLIA_256_CBC_SHA|   & \verb|0x0087| & \no    & \yes & \no  & \no    \\\rowcolor{lightlightgray}
                    & \verb|DHE-DSS-CAMELLIA256-SHA|                        &               & &&&\\
   \bottomrule
    \end{tabular}
\caption{Preference order of cipher suites.  All suites are supported by OpenSSL.}
\label{table:prefOrderCipherSuites}
\end{table}

Note: the above table \ref{table:prefOrderCipherSuites} contains Elliptic curve key exchanges. There are currently strong doubts\footnote{\url{http://safecurves.cr.yp.to/rigid.html}} concerning ECC.
If unsure, remove the cipher suites starting with ECDHE in the table above.


Based on this ordering, we can now define the corresponding settings for servers. We will start with the most common web servers.



\section{Recommendations on practical settings}


\subsection{SSL}
\subsubsection{apache}
\subsubsection{nginx}
\subsubsection{Overview of different SSL libraries: gnutls vs. openssl vs. others}
\subsubsection{openssl.conf settings}

\subsection{SSH}

\subsection{PGP}

\subsection{PRNG settings}

\section{tools}

This section lists tools for checking the security settings.

\subsection{SSL}

ssllabs.com


\url{https://www.ssllabs.com/downloads/SSL_TLS_Deployment_Best_Practices_1.3.pdf}		%% this breaks my pdf converter hmm


\subsection{RNGs}

ent

%havegd



%\section{Further research and unanswered questions}



\chapter{Links}
\label{cha:links}
%% NOTE: this should re restructured...

\begin{itemize*}
  \item IANA official list of Transport Layer Security (TLS) Parameters: \url{https://www.iana.org/assignments/tls-parameters/tls-parameters.txt}
  \item SSL cipher settings: \url{http://www.skytale.net/blog/archives/22-SSL-cipher-setting.html}
  \item Elliptic curves and their implementation (04 Dec 2010): \url{https://www.imperialviolet.org/2010/12/04/ecc.html}
  \item A (relatively easy to understand) primer on elliptic curve cryptography: \url{http://arstechnica.com/security/2013/10/a-relatively-easy-to-understand-primer-on-elliptic-curve-cryptography}
  \item Duraconf, A collection of hardened configuration files for SSL/TLS services (Jacob Appelbaum's github): \url{https://github.com/ioerror/duraconf}
  \item Attacks on SSL a comprehensive study of BEAST, CRIME, TIME, BREACH, LUCKY 13 \& RC4 Biases: \url{https://www.isecpartners.com/media/106031/ssl_attacks_survey.pdf}
  \item EFF How to deploy HTTPS correctly: \url{https://www.eff.org/https-everywhere/deploying-https}
  \item Bruce Almighty: Schneier preaches security to Linux faithful (on not recommending to use Blowfish anymore in favor of Twofish): \url{https://www.computerworld.com.au/article/46254/bruce_almighty_schneier_preaches_security_linux_faithful/?pp=3}
  \item Implement FIPS 183-3 for DSA keys (1024bit constraint): \url{https://bugzilla.mindrot.org/show_bug.cgi?id=1647}
  \item Elliptic Curve Cryptography in Practice: \url{http://eprint.iacr.org/2013/734.pdf}
  \item Factoring as a Service: \url{http://crypto.2013.rump.cr.yp.to/981774ce07e51813fd4466612a78601b.pdf}
  \item Black Ops of TCP/IP 2012: \url{http://dankaminsky.com/2012/08/06/bo2012/}
  \item SSL and the Future of Authenticity, Moxie Marlinspike - Black Hat USA 2011: \url{http://www.youtube.com/watch?v=Z7Wl2FW2TcA}
  \item ENISA - Algorithms, Key Sizes and Parameters Report (Oct.'13) \url{http://www.enisa.europa.eu/activities/identity-and-trust/library/deliverables/algorithms-key-sizes-and-parameters-report}
  \item Diffie-Hellman Groups \url{http://ibm.co/18lslZf}
  \item Diffie-Hellman Groups standardized in RFC3526~\cite{rfc3526} \url{http://datatracker.ietf.org/doc/rfc3526/}
  \item ECC-enabled GnuPG per RFC6637~\cite{rfc6637} \url{https://code.google.com/p/gnupg-ecc}
  \item TLS Security (Survey + Lucky13 + RC4 Attack) by Kenny Paterson \url{https://www.cosic.esat.kuleuven.be/ecc2013/files/kenny.pdf}
  \item Ensuring High-Quality Randomness in Cryptographic Key Generation \url{http://arxiv.org/abs/1309.7366v1}
  \item Wikipedia: Ciphertext Stealing \url{http://en.wikipedia.org/wiki/Ciphertext_stealing}
  \item Wikipedia: Malleability (Cryptography) \url{http://en.wikipedia.org/wiki/Malleability_(cryptography)}
  \item Ritter's Crypto Glossary and Dictionary of Technical Cryptography \url{http://www.ciphersbyritter.com/GLOSSARY.HTM}
\end{itemize*}

\input{suggested_reading}
\section{Reviewers}

We would like to express our thanks to the following reviewers (in alphabetical order):

Horenbeck, Maarten;
Lenzhofer, Stefan;
Schreck, Thomas; 




\bibliography{applied-crypto-hardening}

\end{document}
